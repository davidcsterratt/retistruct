\documentclass{article}

\usepackage{amsmath}

\title{Computing the strain energy of a triangle}
\renewcommand{\vec}[1]{\mathbf{#1}}
\renewcommand{\epsilon}{\varepsilon}

\begin{document}

\section{The problem}
\label{fem:sec:problem}

Suppose we have a triangle whose verticies are at points
\begin{displaymath}
  \vec{p}_i,\quad
  \vec{p}_j,\quad
  \vec{p}_k
\end{displaymath}
and these are moved to 
\begin{displaymath}
  \vec{q}_i=\vec{p}_i+\vec{a}_i,\quad
  \vec{q}_j=\vec{p}_j+\vec{a}_j,\quad
  \vec{q}_k=\vec{p}_k+\vec{a}_k
\end{displaymath}
Assuming that the triangle is isotropic with Young's modulus $E$ and a
Poisson's ratio $\nu$, we wish to compute the strain energy of the
deformed triangle as a function of $\vec{p}_i, \vec{p}_j, \vec{p}_k,
\vec{q}_i, \vec{q}_j$ and $\vec{q}_k$. This will then be differentiated
with respect to $\vec{q}_i, \vec{q}_j$ and $\vec{q}_k$ in order to
determine the force on each node.

\section{Workings so far}
\label{fem:sec:workings-so-far}

Define the matricies:
\begin{displaymath}
  \vec{P} = \left(\vec{p}_i, \vec{p}_j, \vec{p}_k\right) \quad
  \vec{Q} = \left(\vec{q}_i, \vec{q}_j, \vec{q}_k\right)    
\end{displaymath}
Determine the transformation $\vec{M}$ undergone by the triagle:
\begin{displaymath}
  \vec{Q} = \vec{M}\vec{P} \quad \mbox{hence} \quad
  \vec{M} = \vec{Q}\vec{P^{-1}}
\end{displaymath}
Using polar decomposition, factorise $\vec{M}$ into a rotation
$\vec{R}$ and a unitary matrix $\vec{U}$:
\begin{displaymath}
  \vec{M} = \vec{R}\vec{U}
\end{displaymath}
Compute the displacements in the original frame of reference
\begin{displaymath}
  \vec{A} = \vec{R^{-1}}\vec{Q} - \vec{P}
\end{displaymath}
Convert the displancements matrix $\vec{A}$ into a column vector
$\vec{a}$. Find the forces $\vec{f}$ in the original frame using:
\begin{displaymath}
  \vec{f} = \vec{B^T}\vec{D}\vec{B}\vec{a}
\end{displaymath}
Separate forces back into individual forces, and convert to new fram
of reference:
\begin{displaymath}
  \vec{G} = \vec{R}\vec{F}
\end{displaymath}

The original coordinates of a point in the triangle and its
displacement are defined as:
\begin{displaymath}
  \vec{p}= \begin{pmatrix}
    x \\
    y
  \end{pmatrix}
\quad\mbox{and}\quad
\vec{u}= \begin{pmatrix} 
  u \\
  v
\end{pmatrix}
\end{displaymath}
respectively.

The strain associated with that point is defined as:
\begin{displaymath}
  \vec{\epsilon} =
  \begin{pmatrix}
    \epsilon_x \\ \epsilon_y \\ \gamma_{xy}
  \end{pmatrix}
  = 
  \begin{pmatrix}
    \frac{\partial u}{\partial x} \\
    \frac{\partial v}{\partial y} \\
    \frac{\partial u}{\partial y} +
    \frac{\partial v}{\partial u} 
  \end{pmatrix}
\end{displaymath}
The strain is dimensionless.

The stress causing that strain is denoted $\vec{\sigma}$, and is
related to the strain by the elasticity matrix D:
\begin{displaymath}
  \vec{\sigma} =
  \begin{pmatrix}
    \sigma_x \\ \sigma_y \\ \tau_{xy}
  \end{pmatrix}
  = 
  \vec{D}\vec{\epsilon}
\end{displaymath}
For plane, isotropic stress, the elasticity matrix is:
\begin{displaymath}
  \vec{D} = \frac{E}{1-\nu^2}
  \begin{pmatrix}
    1 & \nu & 0 \\
    \nu & 1 & 0 \\
    0   & 0 & (1-\nu)/2
  \end{pmatrix}
\end{displaymath}
The Poisson ratio $\nu$ is dimensionless and Young's modulus $E$ has
units of pressure (Force per unit area).

The strain energy density is
\begin{displaymath}
  \frac{1}{2}\vec{\sigma}^T\vec{\epsilon} = 
  \frac{1}{2}\vec{\epsilon}^T\vec{D}\vec{\epsilon}
\end{displaymath}


\end{document}
