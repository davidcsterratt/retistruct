\documentclass[final,hyperref={pdfpagelabels=false}]{beamer} 
\mode<presentation> {  %% check http://www-i6.informatik.rwth-aachen.de/~dreuw/latexbeamerposter.php for examples
  \usetheme{Sterratt}    %% you should define your own theme e.g. for big headlines using your own logos 
}
% \usepackage[garamond]{mathdesign}
\usepackage[english]{babel}
\usepackage[latin1]{inputenc}
\usepackage[T1]{fontenc}
% \usepackage{amsmath,amsthm, amssymb, latexsym}
\usepackage{helvet}%\usefonttheme{professionalfonts}  % times is
% obsolete
\usepackage[garamond]{mathdesign}
\renewcommand{\rmdefault}{ugm}
\renewcommand{\sfdefault}{ugm}
\newcommand{\figack}[1]{{\par\small\vskip -0.5ex\hfill{\color{blue} #1}\par}}
\usepackage{natbib}
\makeatletter
\def\newblock{\beamer@newblock}
\makeatother 


% \usefonttheme[onlymath]{serif}
\boldmath
\usepackage[orientation=landscape,size=a4,debug]{beamerposter}                       % e.g. for DIN-A0 poster
% \usepackage[orientation=portrait,size=a1,scale=1.4,grid,debug]{beamerposter}                  % e.g. for DIN-A1 poster, with optional grid and debug output
% \usepackage[size=custom,width=200,height=120,scale=2,debug]{beamerposter}                     % e.g. for custom size poster
% \usepackage[orientation=portrait,size=a0,scale=1.0,printer=rwth-glossy-uv.df]{beamerposter}   % e.g. for DIN-A0 poster with rwth-glossy-uv printer check
% ...
% 
\title{Inference of original retinal coordinates from flattened
  retinae} \author{David C Sterratt$^1$ and Ian D Thompson$^2$}
\institute{$^1$Institute for Adaptive \& Neural Computation, School of
  Informatics, University of Edinburgh\\ $^2$MRC Centre for Developmental
  Neurobiology, King's College London} \date{Jul. 31th, 2007}
\begin{document}

\begin{frame}{} 
  \begin{columns}[T]

    %%%%%%%%%%%%%%%%%%%%%%%%%%%%%%%%%%%%%%%%%%%%%%%%%%
    \begin{column}{.33\linewidth}
      \begin{block}{Motivation: Retrograde labelling studies in the the
          developing visual system}

        \includegraphics[width=0.49\linewidth]{../figures/UptoEtal07emer1}
        \includegraphics[width=0.49\linewidth]{../figures/UptoEtal07emer}
        \figack{\citet{UptoEtal07deve}}

        \begin{itemize}
        \item Problem: cell bodies which were neighbours in the original
          retina may be far apart in the flattened retina
        \item Solution: morph the flattened retina back onto a sphere
        \end{itemize}
      \end{block}

    \end{column}

    %%%%%%%%%%%%%%%%%%%%%%%%%%%%%%%%%%%%%%%%%%%%%%%%%%
    \begin{column}{.33\linewidth}

      \begin{block}{Method: Stitching and triangulation}
        \begin{columns}
          \begin{column}{0.4\linewidth}
            \includegraphics[width=\linewidth]{../figures/M634-4-triangulated-stitched2}            
          \end{column}
          \begin{column}{0.55\linewidth}
            \begin{itemize}
            \item Cuts and tears marked up by expert and stitched
              automatically by pairing points at equivalent fractional
              distances along each cut or tear
            \item Delaunay Triangulation of points on retinal outline and
              randomly-generated internal points
            \item Iterative procedure to detect and remove ``flaps''
            \end{itemize}
          \end{column}
        \end{columns}
      \end{block}


      \begin{block}{Method: Initial projection}
        \begin{columns}
          \begin{column}{0.4\linewidth}
            \includegraphics[width=\linewidth]{../figures/M634-4-initial-proj2}
          \end{column}
          \begin{column}{0.55\linewidth}
            \includegraphics[width=0.5\linewidth]{../figures/M634-4-initial-proj-3d2}  

            \begin{itemize}
            \item Project grid onto sphere curtailed at latitude of 50$^\circ$
              \begin{itemize}
              \item Radius determined by area of the flattened retina.
              \end{itemize}
            \item Fix points on rim of flattened retina to the
              latitude of the rim 
            \item Cell bodies projected using barycentric coordinates
            \end{itemize}
          \end{column}
        \end{columns}
      \end{block}

      \begin{block}{Method: Final projection}
        \begin{columns}
          \begin{column}{0.4\linewidth}
            \includegraphics[width=\linewidth]{../figures/M634-4-final-proj2}
          \end{column}
          \begin{column}{0.55\linewidth}
            \includegraphics[width=0.5\linewidth]{../figures/M634-4-final-proj-3d2}

            % \item The aim now is to infer the latitude $\phi_i$ and longitude
            %   $\lambda_i$ on the sphere to which each grid point $i$ on the
            %   flattened retina is projected.

            $E = \frac{1}{2} \sum_{i\in\mathcal{C}} \frac{(l_i -
              L_i)^2}{L_i}  + \alpha\sum_{i\in\mathcal{T}} \exp(-k\frac{a_i}{A_i})
            $
            
            \begin{itemize}
            \item Infer optimal projection onto sphere by changing
              location of vertices on sphere so as to minimise an
              energy function $E$ with two components:
              \begin{itemize}
              \item Sum of squared differences between lengths of corresponding
                connections $\mathcal{C}$ in flattened retina $L_i$ and on sphere $l_i$
              \item Sum of exponentials of ratios of areas  of 
                triangles $\mathcal{T}$ in flattened retina $A_i$ and  on sphere $a_i$
              \end{itemize}
            \end{itemize}
          \end{column}
        \end{columns}
      \end{block}

%       \begin{block}{Problems}
%         \begin{itemize}
%         \item Method for folding flattened retina onto hemisphere has
%           been developed 
%         \end{itemize}
%       \end{block}


    \end{column}

    %%%%%%%%%%%%%%%%%%%%%%%%%%%%%%%%%%%%%%%%%%%%%%%%%% 
    \begin{column}{.33\linewidth}

      \begin{block}{Validation}
        \begin{columns}
          \column[T]{0.7\linewidth}

          \includegraphics[width=0.33\linewidth]{../data/14-12-09_1754.jpg}
          \includegraphics[width=0.33\linewidth]{../data/14-12-09_1755.jpg}
          \includegraphics[width=0.33\linewidth]{../data/14-12-09_1758.jpg}

          \includegraphics[width=0.33\linewidth]{../figures/orange-no-grid.png}
          % \includegraphics[width=0.33\linewidth]{../figures/orange-outline.pdf}
          % \includegraphics[width=0.33\linewidth]{../figures/orange-stitched.pdf}
          % \includegraphics[width=0.33\linewidth]{../figures/orange-stitched-triangulated.pdf}
          \includegraphics[width=0.33\linewidth]{../figures/orange-many-grid.pdf}
          \includegraphics[width=0.33\linewidth]{../figures/orange-grid.png}


          \column[T]{0.25\linewidth}
          % \includegraphics[width=\linewidth]{../figures/orange-projected.pdf}
          \includegraphics[width=\linewidth]{../figures/orange-final.pdf}

        \end{columns}
      \end{block}

      \begin{block}{Discussion \& Conclusions}
        \begin{itemize}
        \item Method for folding flattened retina onto partial sphere has
          been developed, and validation study suggests it is
          reasonably accurate.
        \item Allows for more quantitative indication of location of
          cell bodies  
        \item At present, the method needs supervision to indicate where cuts
          and tears are
        \item Possible unsupervised alternative approach to stitching:
          attraction between edges
          \begin{itemize}
          \item This was tried, but proved complex to implement in a
            way such that the correct mappings were made
          \end{itemize}
        \item Applications
          \begin{itemize}
          \item Statistics of retrograde tracing
          \item Allowing electrode stimulation arrays to map activity
            in more realistic fashion onto flattened retina
          \end{itemize}
        \item Next steps:
          \begin{itemize}
          \item Test on more data
          \item Speed up the optimisation procedure.
          \end{itemize}
        \end{itemize}
      \end{block}

      \begin{block}{Acknowledgements}

        KCL: Daniel Nedergaard, Ian Thompson, Andrew Lowe

        ANC: Michael Herrmann, David Willshaw, Matthias Hennig

        This work was funded by the Wellcome Trust.
      \end{block}


      \begin{block}{References}
        \footnotesize
        \bibliographystyle{apalike}
        \bibliography{mystrings,my}
      \end{block}
    \end{column}


  \end{columns}



\end{frame}


\end{document}
