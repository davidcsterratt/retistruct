\documentclass{book}

\newcommand{\svninfo}{Retistruct version: 0.6.2 , Date: 2019-12-13}
\pagestyle{myheadings}
\markboth{\svninfo}{\svninfo}

\usepackage[a4paper,left=1in,right=1in,top=1in,bottom=1in,head=1in]{geometry}
\usepackage{mathpazo}
\usepackage[british]{babel}
\usepackage{tabularx}
\usepackage{graphicx}
\usepackage{hyperref}
\usepackage{gensymb}
\usepackage{titlesec}

\newenvironment{aside}[1]%
{\noindent\begin{minipage}{\textwidth}\vspace*{1em}\hrule height2pt\vspace*{2pt}
  \noindent{\bf Note: #1}\par%
  \vspace*{2pt}%
  \hrule height1pt
  \vspace*{2pt}\noindent\ignorespaces}%
  {\vspace*{2pt}\hrule height1pt\vspace*{1em}\end{minipage}}

\titleformat{\section}{\hrule\normalfont\Large\bfseries}{\thesection}{1em}{}
\titleformat{\subsection}{\hrule\normalfont\large\bfseries}{\thesubsection}{1em}{}

\title{\textsc{Retistruct} manual}
\author{David C. Sterratt}

\begin{document}

\maketitle
\thispagestyle{myheadings}

\tableofcontents

\chapter{Installation}
\label{manual:sec:installation}

\section{Install the necessary system packages}

\subsection{Ubuntu Linux}

Follow the instructions at
\url{https://cran.r-project.org/bin/linux/ubuntu/}
to obtain the latest stable version of R. For example, for Ubuntu
16.04 (Xenial) to download R from the Bristol UK mirror, enter the
following in a terminal window:
\begin{verbatim}
sudo apt-key adv --keyserver keyserver.ubuntu.com --recv-keys E084DAB9
sudo apt-add-repository deb https://https://www.stats.bris.ac.uk/R/bin/linux/ubuntu xenial/
sudo apt-get install r-base r-base-dev libgtk2.0-dev libgl1-mesa-dev libglu1-mesa-dev
\end{verbatim}
% Need to install RGtk2 separately??

\subsection{Fedora Linux }

\emph{NOTE: these instructions have not been tested since at least
  2015.}

As root,

\begin{verbatim}
yum install R-core
yum install R-devel
yum install gtk2
yum install gtk2-devel
yum install mesa-libGL-devel
yum install mesa-libGLU-devel
\end{verbatim}


\subsection{Mac OS X}
\label{retistruct-manual:sec:mac}

First install R for Mac OS X, available from
\url{http://www.r-project.org/}. Under Mac OS X Leopard and Lion, this
should be sufficient. 

\subsubsection{Under Mac OS X 10.8 (Snow Leopard),} it may be necessary to
install the following external packages:
\begin{itemize}
\item Xcode (free from App Store)
\item Xquartz (\href{http://xquartz.macosforge.org/}{http://xquartz.macosforge.org/})
\item GTK (\href{http://r.research.att.com/libs/GTK_2.18.5-X11.pkg}{http://r.research.att.com/libs/GTK\_2.18.5-X11.pkg})
\end{itemize}

\subsubsection{Under Mac OS X 10.9 (Mavericks) or Mac OS X 10.10
  (Yosemite),}
See this issue on Github: \url{https://github.com/davidcsterratt/retistruct/issues/4}

\subsection{MS/Windows}

First install R for MS/Windows, available from \url{http://www.r-project.org/}.

\section{Install the core \textsc{Retistruct} package}

\begin{enumerate}
\item Start \textsc{R}
 \item  Type:
\begin{verbatim}
install.packages("retistruct")
\end{verbatim}
   The first time this runs, it should create a personal directory for
   R packages, and it will take a few minutes to install some required
   packages.
 \item Test that it is working as follows (from within R)
\begin{verbatim}
   library(retistruct)
   demo("retistruct.method")
\end{verbatim}
   If all is well, you should be prompted to press \textbf{Return} a
   few times and Figure 1 from the paper \cite{SterrattEtal2012} will
   appear (drawing the 3D renditions takes some time). There will be
   reports of warnings; they are to do with printing the figures to
   files. You can view them by typing \texttt{warnings()}.
 \item A further test produces Figure~2:
\begin{verbatim}
   demo("low.high")
\end{verbatim}
 \item A further (quite long-running) test produces Figure~6:
\begin{verbatim}
   demo("figure6")
\end{verbatim}
\end{enumerate}

\section{Install the \textsc{Retistruct} GUI}
\label{retistruct-manual:sec:inst-retistr-gui}

If the previous step works:
\begin{enumerate}
\item Type:
\begin{verbatim}
retistruct()
\end{verbatim}
\item Some more packages will be downloaded -- this may take a
  minute or so.
\item If you are using MS/Windows, there may be an error message
  saying that libatk-1.0-0.dll is missing from your computer. Click on
  \textbf{OK}. Another message will appear, asking if you want to
  install GTK+. Agree to this.  After this has finished, quit R
  (without saving the workspace image) and restart.
\item If all works, an interface window should appear. A number of
  sets of demonstration data are available from the \textbf{Demo} menu item.
\end{enumerate}

% \section{A tutorial}
% \label{retistruct-user-guide:sec:tutorial}

% This is a tutorial using some sample data provided on the website. 

% \begin{enumerate}
% \item Download the data from
%   \href{http://retistruct.r-forge.r-project.org/Figure_6-data.zip}{http://retistruct.r-forge.r-project.org/Figure\_6-data.zip}. This
%   is the data that underlies Figure~6 of \cite{SterrattEtal2012}
% \item Unpack the zip file. A directory \texttt{Figure\_6-data} will be created.
% \item Start retistruct by first starting \textsc{R}. At the \textsc{R}
%   prompt type:
% \begin{verbatim}
% > library(retistructgui)
% > retistruct()
% \end{verbatim}
%   A window should appear.
% \item Click on the \textbf{Open} icon and navigate to the
%   \texttt{left-contra} directory in the \texttt{Figure\_6-data} directory.
% \end{enumerate}

\chapter{Running \textsc{Retistruct}}
\label{manual:sec:running}

To start the program, start \textsc{R}. At the \textsc{R} prompt type:

\begin{verbatim}
library(retistruct)
retistruct()
\end{verbatim}
A window should appear.

\section{Preparing and opening the files for a retina}
\label{manual:sec:opening-files-retina}

There are a number of types of information associated with a
flat-mount retina that \textsc{Retistruct} can process:
\begin{itemize}
\item The coordinates of the outline of the flattened retina
\item An image of the flat-mount retina (optional)
\item The coordinates of labelled data points within the flat-mount
  retina (optional)
\item The coordinates of labelled counts of data within the flat-mount
  retina (optional)
\item The scale of the coordinates provided in the other files
  (optional)
\end{itemize}
To import this information into \textsc{Retistruct}, for each retina,
a directory should be created containing files with the above
information, as will be described below.  At present there are three
formats of directory that \textsc{Retistruct} can read.  Most users
will probably find the the \textsc{ImageJ} ROI format most convenient.

\section{Reading files using the \textsc{ImageJ} ROI format}
\label{retistruct-manual:sec:ijroi-format}

This format allows you to load images of retinae whose outlines have
been marked up in \textsc{ImageJ}. To create the image and outline
files for this format:
\begin{enumerate}
\item Create a directory to save the files created to.
\item Open up \textsc{ImageJ} (or FIJI).
\item Use \textbf{File$\rightarrow$Open} to open the image.
\item Use \textbf{Image$\rightarrow$Scale} to down sample so that the
  resolution is less than 1000x1000. (This is not crucial, but will
  speed up things later.)
\item Save the down-sampled version of the image to the file name
  \texttt{image.png} in the directory created in Step 1.
\item Use the Polygon Tool
  (\includegraphics[height=\baselineskip]{poly}) to mark the edge of
  the retina. According to the \textsc{ImageJ}
  manual\footnote{http://rsbweb.nih.gov/ij/docs/tools.html}: ``To
  create the selection, click repeatedly with the mouse to create line
  segments. When finished, click in the small box at the starting
  point (or double-click), and ImageJ automatically draws the last
  segment.

  The points that define a polygon selection can be moved or deleted,
  and new points can be added. To delete a point, click on it with the
  alt key down. To add a point, click on an existing point with the
  shift key down.''
\item Open the ROI manager by selecting
\textbf{Analyze$\rightarrow$Tools$\rightarrow$ROI Manager}.
\item Click on the \textbf{Add [t]} button.
\item Click on the \textbf{More$\rightarrow$Save} button. In the \textbf{Save
selection\dots} box that appears, enter the file name
\texttt{outline.roi} and make sure this file is saved to the same
directory as the \texttt{image.png} file, i.e.\ the directory created
in Step 1.
\item Optionally mark up the optic disc in the same way as the
  outline. Save this ROI to a file called \texttt{od.roi} in the same
  directory as \texttt{image.png}.
\item Open \textsc{Retistruct}.
\item Click on the \textbf{Open} icon and select the directory
containing \texttt{image.png} and \texttt{outline.roi}. The retinal
image should now appear, with the outline shown. By default the
outline is in black. If this isn't visible against the image, press
the \textbf{Properties} button in the interface (or select \textbf{Edit$\rightarrow$Properties}), and change the
\textbf{Outline colour}.
\end{enumerate}

\subsection{Specifying the scale} Optionally, the scale of the image
can be specified by providing a file \texttt{scale.csv} in the same
directory as the image and ROI files. This file should contain two
lines, the first with the heading \texttt{Scale} and the second with
the length of the side of a pixel in micrometres. For example:
\begin{verbatim}
"Scale"
1.5
\end{verbatim}

\subsection{Reading in data points}
\label{retistruct-user-guide:sec:read-data-points}

In general, the coordinates of data points are read in from a csv file
called \texttt{datapoints.csv}. The two cells of the first line of the
file contain the name of the group of data points and the colour that
these should be displayed in \textsc{Retistruct}.  Marking up the
points from the image in \textsc{ImageJ} will ensure that the
coordinates are in the same system as those used for the image and the
outline. To do this:
\begin{enumerate}
\item Open the image used for marking up the outline
  (\texttt{image.png}) in \textsc{ImageJ}.
\item Threshold the image
  (\textbf{Image$\rightarrow$Adjust$\rightarrow$Threshold\dots} so
  that the labelled points are visible.
\item Select \textbf{Analyze$\rightarrow$Set Measurements\dots} and make
  sure that \textbf{Centroid} is checked
\item Select \textbf{Analyze$\rightarrow$Analyze Particles}. Make sure
  that \textbf{Display results} and \textbf{Clear results} are
  checked. Click on \textbf{OK}.
\item A window entitled \textbf{Results} should appear. This contains
  the X and Y coordinates of the detected points. In this window
  select \textbf{File$\rightarrow$Save as\dots} and save the file as
  \texttt{datapoints.csv}.
\item Open this file with either \textsc{Excel} or \textsc{OpenOffice
    Calc} and remove all the columns apart from the X and Y
  columns. Replace the X with the name of the data set and the Y with
  the colour it should be displayed in \textsc{Retistruct}.
\item Save \texttt{datapoints.csv} in the retina's directory, along
  with \texttt{outline.roi} (and \texttt{image.png} if an image is
  desired).
\end{enumerate}

\subsection{Reading in data counts}
\label{retistruct-user-guide:sec:read-data-counts}

In general, the coordinates of data counts (counts of data at X, Y
positions) are read in from a csv file called \texttt{datacounts.csv}.

The two cells of the first line of the file contain the name of the
group of data counts and the colour that these should be displayed in
\textsc{Retistruct}. 

Each subsequent row has three numbers: The first two columns contain
the X and Y coordinates at which the count was measured, and the third
column contains the count at that location. It is important to ensure
that the coordinates are in the same system as those used for the
image and the outline.

Save \texttt{datapoints.csv} in the retina's directory, along with
\texttt{outline.roi} (and \texttt{image.png} if an image is desired).

\section{Reading in files using other formats}
\label{retistruct-user-guide:sec:reading-files-using}

\subsection{CSV format}
\label{retistruct-manual:sec:csv-format}

This is the same as the \textsc{ImageJ ROI} format, except that the
outline is contained in the two columns of a file called
\texttt{outline.csv}.  Each column should have a heading, e.g.:
\begin{verbatim}
"X", "Y"
1,5
10,76,
...
\end{verbatim}
The optic disc outline can be supplied in a similar file called
\texttt{od.csv}.

\subsection{IDT format}
\label{retistruct-manual:sec:idt-format}

This is the format used in Ian Thompson's lab (see
Appendix~\ref{manual:sec:reading-data}). These files are contained in
a directory. To open the files corresponding to a retina, click on the
open file icon, and navigate to the directory containing the
\texttt{SYS} and \texttt{MAP} files. On opening this directory, the
retinal outline should appear in the \textsc{Retistruct} window.

\section{Editing the retinal mark-up}
\label{manual:sec:marking-up-retina}

After a set of files is opened, the \textbf{Edit} tab in the
\textsc{Retistruct} window will be open. Before the retina can be
reconstructed, you need to mark-up tears and landmarks on the retina
using the following controls:

\begin{description}
\item[Add tear] To add a tear, click on this button, then click on
  three points in turn which define a tear. The order in which the
  points are added does not matter. Tears contained within a tear can
  be marked up, but tears cannot cross over one another.
\item[Move Point] To move one of the points defining a tear, click on
  this button, then click on the point which you desire to move, then
  click on the point to which it should be moved.
\item[Remove tear] To remove a tear, click on this button, then click
  on the apex of the tear (marked in cyan on the plot)
\item[Mark nasal] To mark the nasal pole, click on this button, then
  click on the point which is the nasal pole.  If the nasal or
  dorsal pole has already been marked, the marker is removed from
  the existing location. The nasal pole should not be in a tear. If
  the nasal tear is placed within a tear, no error is reported at this
  stage, but it will be reported later.
\item[Mark dorsal] As above, except for the dorsal pole.
\item[Mark OD] To mark the optic disc, click on the structure marked
  in orange which you think is the OD. Once clicked on, the structure
  should become blue.
\item[Phi0] This determines the latitude of the rim of the
  reconstructed hemisphere. In mouse, it depends on the age of the
  animal \cite{SterrattEtal2012}.
  \textbf{Ideally you will have measured or estimated this value from
    an intact retina.}
  The default value of 0 may lead to reconstructions that are not as
  accurate as they could be, since the template that
  \textsc{Retistruct} is trying to morph the flat-mount retina onto is
  not the same as the original shape.
\end{description}

\section{Editing metadata}
\label{retistruct-manual:sec:metadata}

You can also edit the following metadata:

\begin{description}
\item[Data/Flip DV] Flip the DV axis to compensate for microscope
  orientation. Affects display of retinae.
\item[Eye] Specify whether the eye is Right or Left.  Affects display
  of retinae.
\end{description}

\section{Saving the markup and metadata}
\label{manual:sec:saving-markup}

To save the markup, click on the ``Save'' button in the toolbar. This
saves various markup files to the directory containing the
data files. This saved data can be used to reconstruct the retina using
a batch process (Section~\ref{manual:sec:runn-batch-reconstr}).

\section{Reconstructing the retina}
\label{manual:sec:reconstr-retina}

To reconstruct the retina, click on the ``Reconstruct retina'' button.
This causes \textsc{Retistruct} to perform a (sometimes lengthy)
sequence of operations:
\begin{description}
\item[Stitching] Links between corresponding points on parts of the retinal
  outline  contained in tears are made.
\item[Triangulation] A triangular mesh is placed over the flattened retina
\item[Initial projection to sphere ] The mesh is projected roughly
  onto a sphere
\item[Optimisation] The locations of the mesh points on the hemisphere
  is adjusted so as to minimise a weighted sum of the squared
  differences between the lengths of links in the mesh on the
  hemisphere and on the flattened retina, whilst ensuring that as few
  triangles as possible are flipped.
\end{description}

At the end of the reconstruction process, a polar plot appears next to
the flattened retina. By default the location of
the cuts and tears in the polar coordinates can be seen.

\section{Assessing the quality of the reconstruction}
\label{retistruct-user-guide:sec:assess-qual-reconstr}

It is important to assess the quality of the reconstruction. The
choice of where to mark up tears can affect this adversely, as can the
rim angle.

One diagnostic is to view the strain in each mesh link by clicking on
the \textbf{Strain} checkbox in the \textbf{Edit} tab. When \textbf{Strain}
is shown, the flat plot shows the strain in the links of the mesh
using a colour code from blue (compressed) to red (expanded). The
polar plot is replaced by a scatter plot of the length of links in the
reconstructed object versus the length on the flattened object. The
flat plot of a good reconstruction will look mostly green, and in the
scatter plot the points will lie close to the black line.

\section{Selecting viewing options}
\label{retistruct-user-guide:sec:select-view-opti}

To select viewing options click on the \textbf{View} tab. A number of
options are now shown, grouped into a number of sections:

\subsection{Show} In the \textbf{Show} section there are checkboxes
that allow you to show various types of information:
\begin{description}
\item[Markup] Locations of tears and the dorsal or nasal pole
\item[Stitch] Locations of how the algorithm has stitched tears (only
  visible after the reconstruction step)
\item[Grid] Lines of latitude and longitude projected back onto the
  flattened retina (only visible after the reconstruction step) and
  grid lines are shown on the reconstructed retina
\item[Landmarks] Landmarks such as the optic disc
\item[Points] Locations of data points, such as the locations of beads
  of dye, which have been imported into \textsc{Retistruct} as
  described in Section~\ref{retistruct-user-guide:sec:read-data-points}
\item[Point means] Locations of the Karcher mean (mean in spherical
  coordinates) of each group of data points
\item[Point contours] Kernel Density Estimate contours of the data
  points. By default, contours at 5\%, 25\%, 50\%, 75\% and 95\% of
  the maximum density are shown. This can be changed by a command like
  this at the R prompt:
\begin{verbatim}
options(contour.levels=c(5, 25, 50))
\end{verbatim}
  This command would cause the 5\%, 25\% and 50\% contour lines to be
  displayed.
\item[Counts] Counts (for example of cells), represented by a number
  at the location at which the count was made, which have been
  imported into \textsc{Retistruct} as described in
  Section~\ref{retistruct-user-guide:sec:read-data-counts}
\item[Count contours] Kernel Regression contours of the data counts.
  By default, contours at 5\%, 25\%, 50\%, 75\% and 95\% of the
  maximum density are shown. This can be changed by a command, as
  shown under ``Point contours''.
\end{description}

\subsection{IDs}

If you included a \texttt{datapoints.csv} and/or a
\texttt{datacounts.csv} file (see
Sections~\ref{retistruct-user-guide:sec:read-data-points}
and~\ref{retistruct-user-guide:sec:read-data-counts})
the \textbf{IDs} section will contain checkboxes with names
corresponding to the names of the groups of points and the names of
the groups of counts specified in those files. If you have marked the
optic disc, it will also contain an \textbf{OD} checkbox. Check or
uncheck these boxes to show or hide the corresponding points and
counts.

\subsection{Projection}
\label{retistruct-user-guide:sec:projection}

The reconstructed data can be viewed in a number of projections, just
as there are various ways that the surface of the globe is projected
to a 2D image. Because the geometry of the surface of a sphere is
fundamentally different from a flat sheet of paper, no projection is a
perfect representation; all have advantages and disadvantages.

\begin{aside}{Coordinate systems}
  The co-ordinates of a point on the reconstructed sphere can be
  described in two ways:
  \begin{description}
  \item[Latitude and longitude] Here latitude and longitude are
    defined analogously to the standard geographical coordinate
    system. We imagine the retina lying with its optic pole at the
    south pole, and with its widest point lying on the equator. The
    latitude of the optic pole is $-90\degree$ and the latitude of
    point on the widest point of the retina (its ``equator'') is
    $0\degree$. We denote the latitude $\varphi$ and the longitude
    $\lambda$.
  \item [Colatitude and longitude] Here we imagine the retina in the
    same orientation with respect to the earth, but we rather than
    latitude, we use colatitude measured from the south pole, i.e. the
    south pole has a colatitude of $0\degree$ and the widest point
    (the ``equator'') has a colatitude of $90\degree$. We denote
    colatitude by $\psi$ and it is related to latitude by
    $\psi=\varphi+90\degree$.
  \end{description}

\end{aside}

Currently the options are:
\begin{description}
\item[Azimuthal Equidistant] The default projection. This is a polar
  plot in which the radial distance is proportional to the colatitude
  (latitude measured from the ``South Pole''; see note on coordinate
  systems), and angle from the horizontal is equal to the longitude on
  the original retina. Points equidistant on line of longitude are
  also equidistant in this projection. It is not an area-preserving
  projection: it makes regions closer to the rim look bigger than they
  actually are relative to areas close to the south pole.
\item[Azimuthal Equal Area] In this projection, also known as a
  Lambert projection, %\cite{FisherEtal1987}
  points that are
  equidistant on lines of longitude on the spherical retina are not
  equidistant: points close to the rim are mapped to be closer
  together. This is done so that area is preserved in the sense that
  equal areas on the sphere project to equal areas on the plane of
  projection. It is thus an area-preserving projection.
\item[Azimuthal Conformal] In this projection, also known as a Wulff
  projection, angles are preserved. This is done at the expense of
  expanding the representation of areas close to the rim.
\item[Sinusoidal] The sinusoidal projection projects the entire globe
  onto the plane and preserves area. The user selects the longitude at
  the centre of the projection in the \textbf{Az} box in the
  \textbf{Projection centre} section of the window. The sinusoidal
  projection is helpful for data transformed into visuotopic space
  (see later).
\item[Orthographic] The orthographic projection gives a perspective
  view of one side of the globe. The projection is centred on a
  latitude and longitude specified by the values in degrees in the
  \textbf{El} and \textbf{Az} boxes in the \textbf{Projection centre}
  section. % \cite{WolframOrthographic2012}.


\end{description}

\begin{table}
  \centering
  \begin{tabularx}{1.0\linewidth}{lll}
    \hline
    \textbf{Projection}
    & \textbf{Latitude \& longitude} 
    & \textbf{Colatitude \& longitude} \\
    \hline
    Azimuthal Equidistant 
    & $\rho = 90\degree +\varphi$, $\lambda=\lambda$
    & $\rho = \psi$, $\lambda=\lambda$ \\
    Azimuthal Equal Area 
    & $\rho =\sqrt{2(1+\sin \varphi )}$, $\lambda=\lambda$
    & $\rho =\sqrt{2(1-\cos \psi )}$, $\lambda=\lambda$ \\
    Azimuthal Conformal 
    & $\rho =\tan\left(45\degree +\varphi /2\right)$,
      $\lambda=\lambda$
    & $\rho =\tan\left(\psi/2\right)$,
      $\lambda=\lambda$  \\
    Sinusoidal
    & $x=(\lambda -\lambda _0)\cos
      \varphi$, $y=\varphi $
    &
    \\
    Orthographic
    & $x=\cos \varphi \sin
      (\lambda -\lambda _0)$
    & \\
    & $y=\cos \varphi _0\sin \varphi -\sin \varphi
      _0\cos \varphi \cos (\lambda -\lambda _0)$ \\
    \hline
  \end{tabularx}
  \caption{Projections. For the three azimuthal projections, the 
    polar coordinates $(\rho, \lambda)$ are given in terms of the
    latitude and longitude $(\varphi,\lambda)$ or colatitude and
    longitude $(\psi,\lambda)$ on the original sphere. For the
    sinusoidal and orthographic projection the $(x,y)$ coordinates in
    the plane are given. For these projections
    $(\varphi_0,\lambda_0)$ is the latitude and longitude at 
    the centre of the projection.}
  \label{tab:projections}
\end{table}

% is given by $x=(\lambda -\lambda _0)\cos \varphi ,y=\varphi $, where
% \cite{WolframSinusoidal2012}.

% The data on the sphere can be projected onto a polar or azimuthal
% equidistant plot in the plane with coordinates  $(\rho ,\lambda )$ 
% where the radius  $\rho =\pi /2+\varphi $. 

\begin{aside}{Convention on longitude of nasal, dorsal, temporal and
    ventral poles}
By convention, the polar plots are viewed as though the animal is
facing towards the observer. This means that when plotting a retina
from a right eye, the nasal pole on the right and N, D, T, V are in
anticlockwise order; for a retina from a left eye, nasal is on the
left and N, D, T, V are in clockwise order. The longitude of a point
was defined so that 0$^{\circ}$ is always at the right of the plot and
90$^{\circ}$ at top. This means that for a right eye the poles correspond
to the longitudes as follows: N, 0$^{\circ}$; D, 90$^{\circ}$; T, 180$^{\circ}$;
V, 270$^{\circ}$. For a left eye the longitudes of the nasal and temporal
poles are interchanged: N, 180$^{\circ}$; D, 90$^{\circ}$, T, 0$^{\circ}$; V,
270$^{\circ}$.
\end{aside}

\subsection{Transform}
\label{retistruct-user-guide:sec:transform}

The \textbf{Transform} option allows the reconstructed retina to be
projected into visual space. The options are

\begin{description}
\item[None (default)] No transformation is made.
\item[Invert] The image is inverted as though by a pinhole lens, a
  crude approximation to the optics of the eye. This allows visual
  space to be mapped onto the retina via the optical system of the
  eye. This mapping depends on the angle that a ray makes with the
  optic axis (the \textbf{Axis direction}) which we define as being
  oriented at azimuth (\textbf{Az}) and elevation
  (\textbf{El}). % $\theta _0$ $\alpha _0$
\item[Invert to hemisphere] Similar to \textbf{Invert}, except that
  the projection is compressed or expanded so that the field of view
  is 180\degree.
\end{description}

There are more details of the transformation in the supplemental
information of \cite{SterrattEtal2012}.

% In previous work of the mapping of visual space on the superior
% colliculus \cite{DragerHubel1976}, visual space has been described
% in terms of azimuth  $\theta $  and elevation  $\alpha $   with
% respect to the long axis of the mouse. Visual space is mapped onto
% the retina via the optical system of the eye. This mapping depends on
% the angle that a ray makes with the optic axis, which we define as
% being oriented at azimuth  $\theta _0$   and  elevation  $\alpha _0$.

% In order to map the eye onto visual space, we consider a reference
% frame in which the $z${}-axis
% is vertical, the $x${}-axis
% is pointing forwards along the long axis of the mouse and the
% $y${}-axis
% is pointing rightwards, as viewed by an observer facing the animal.
% Notionally, the optic axis of the eye starts off vertical, and a
% sequence of rotations is undertaken in order to move the eye to its
% correct orientation. First the projection of rays from the retina to
% visual space is determined using the approximate optics defined in the
% main text. Second, the coordinates of points are converted to
% Cartesian coordinates. Third, the coordinates are rotated about the
% $x$-axis by $-90+\alpha _0$ degrees. This ensures that the optic axis
% is oriented at the elevation $\alpha $ above the horizontal, but the
% direction of its azimuth will be $-90^{\circ }$. The points on the eye
% are then rotated about the original $z${}-axis
% by $90+\theta _0$ degrees giving the optic axis an azimuth of $\theta
% _0$, where positive values are to the right, as viewed from a point on
% the positive $x${}-axis.
% Finally the Cartesian coordinates are transformed back to spherical
% coordinates in the original reference frame.


\section{Exporting the plots}
\label{retistruct-user-guide:sec:exporting-plots}

The \textbf{Bitmap} and \textbf{PDF} buttons above the flatmount view
and the reconstructed view export the image show to a PNG file format
or an PDF respectively. To change the width of the image in pixels,
select \textbf{Properties} and set the
\textbf{Maximum width of projection} to the desired number of pixels.
\textbf{Warning:} setting a high number of pixels will create a
high-resolution image, but can also take some minutes.

\section{Saving the reconstruction}
\label{manual:sec:saving-reconstr}

To save the markup, click on the ``Save'' button in the toolbar. This
saves various files to the directory containing the data files. When
the files in the directory are opened again, all the markup
information (cut locations, location of nasal point and rim angle) is
loaded, and the reconstruction will also appear, unless there has been
a major upgrade of the software, in which case the retina will need to
be reconstructed using the ``Reconstruct retina'' button.

\chapter{Further topics}
\label{retistruct-user-guide:cha:advanced-topics}

\section{Accessing reconstruction data from R}
\label{retistruct-user-guide:sec:reading-date-into}

It can be useful to have access to the data underlying a
reconstruction directly, for example to allow statistical analysis.
To load saved date into R, type the following into R:
\begin{verbatim}
r <-
retistruct.read.recdata(list(dataset="/path/to/reconstruction/directory"),
check=FALSE)
\end{verbatim}
The resulting object \texttt{r} contains various fields, which can be
accessed using the R \texttt{\$} operator, e.g.
\begin{verbatim}
r$featureSets[[1]]$Ps
\end{verbatim}
gives a list of the spherical co-ordinates of labelled data points.
Other quantities (for example the locations of the mesh points in the
flat and spherical retina) are also available.

All classes are fully documented - type `help.start()` and browse to
the \texttt{retistruct} package. Then the \texttt{RetinalReconstructedOutline}
class is a good place to start.

% See Table~\ref{tab:matlab-export} for a list of the most useful
% quantities.

% The documentation for \texttt{ReconstructedOutline}, \texttt{getDss}
% and \texttt{getSss} in the user manual has more information about
% this.

\section{Reading reconstruction data into \textsc{Matlab}}
\label{retistruct-manual:sec:export-reconstr-data}

By default when a reconstruction is saved, a subset of the data is
stored in a file called \texttt{r.mat} in the same directory as the
raw data and the markup. To import this data into \textsc{Matlab}, cd
into that directory, and type:
\begin{verbatim}
clear
load r.mat
\end{verbatim}
This puts a number of variables into the workspace, as shown in
Table~\ref{tab:matlab-export}.

To produce a polar plot of data points, try the following code:
\begin{verbatim}
polar(Dss.green(:,2), Dss.green(:,1)*180/pi+90, '.g')
hold on
polar(Dss.red(:,2),   Dss.red(:,1)*180/pi  +90, '.r')
hold off
\end{verbatim}
The radial axis indicates the latitude in degrees measured from the
retinal pole.

In the \texttt{matlab} subdirectory of the distribution, there are
some scripts to produce polar polar plots, including the locations of
tears and landmarks. To create PDF plots of all the retinae in a
directory, try:
\begin{verbatim}
makefigures('retinae', 'output_directory')
\end{verbatim}

There are also some embryonic scripts to create polar plots:
\texttt{plot\_datapoints\_polararea.m} and \texttt{polararea.m}.


\begin{table}[p]
  \begin{tabularx}{\linewidth}{p{1.2in}X}
    \hline \texttt{phi0} & The latitude of the rim, expressed in
    degrees
    \\
    \texttt{Dss} & Structure containing locations of labelled cell bodies in spherical coordinates. The first column
    contains the latitude of each point, measured in radians. The
    second column contains the
    longitude of each point, measured in radians. \\
    \texttt{DssMean} & Location of Karcher
    mean of green-labelled cell bodies in spherical coordinates. The
    first column contains the latitude of each point, measured in
    radians. The second column contains the
    longitude of each point, measured in radians. A value of
                            $-2147483648$ corresponds to NA in R. \\
    \texttt{DssHullarea}, & Structure containing area of
    convex hull of points on sphere. The convex hull is
    essentially a polygon drawn around data points. A value of
                            $-2147483648$ corresponds to NA in R. \\
    \texttt{Tss} & A structure in which each element contains
    spherical coordinates (in the same latitude-longitude format as
    above) of a tear. \\
    \texttt{Sss} & A structure in which each element contains
    spherical coordinates (in the same latitude-longitude format as
    above) of a landmark. \\
    \texttt{KDE} & An object containing information about Kernel
    Density Estimates of the locations of cell bodies.\\
    & \begin{tabular}{p{1.5in}p{3in}} \texttt{green\_flevels} &
      Contour heights, determined by finding heights that exclude a
      certain fraction of the probability. For example, the 95\%
      contour is excludes 95\% of the probability mass,
      and it should enclose about 5\% of the points. \\
      \texttt{green\_labels} & Contour labels. These give the label
      (e.g.\ 5, 25, 50, where these are the percentages above) of each
      contour. Note that there may be more than one contour at the
      same level, so this vector may contain more elements than
      \texttt{flevels}. The first element of \texttt{green\_labels}
      labels the contour whose coordinates are specified in
      \texttt{green\_contours1}, the second element of
      \texttt{green\_labels} relates to \texttt{green\_contours2} and
      so on. \\
      \texttt{green\_tot\_contour\_areas} & The total area in square
      degrees enclosed by each contour. This is a matrix with the
      first column giving the contour label (see above) and the
      next column giving the area. \\
      \texttt{green\_kappa} & The concentration parameter of
      the Fisher density determined by the kernel fitting algorithm. \\
      \texttt{green\_h} & A pseudo-bandwidth parameter, the inverse
      of the square root of \texttt{kappa}. Units of degrees. \\
      \texttt{green\_maxs\_phi} & Lattitude of maximum point of kernel
      estimate. \\
      \texttt{green\_maxs\_lambda} & Longitude of maximum point of kernel
      estimate. \\
      \texttt{green\_g\_xs} \texttt{green\_g\_ys} \texttt{green\_g\_f}
      & Kernel density estimates on standard polar grid. This can be
      plotted in MATLAB using the command
      \texttt{contour(KDE.green\_g\_xs, KDE.green\_g\_ys,
        KDE.green\_g\_f)}. \\
      \texttt{green\_gpa\_xs} \texttt{green\_gpa\_ys}
      \texttt{green\_gpa\_f} & Kernel density estimates on
      area-preserving polar grid. Plotted in MATLAB as above. \\
      \texttt{green\_contours1}, \texttt{green\_contours2\dots} &
      Coordinates of contours. See \texttt{green\_labels} above for
      more explanation. \\
      \texttt{green\_contour\_areas1},
      \texttt{green\_contour\_areas2\dots} & Area contained within each
      individual contour. See \texttt{green\_labels} above for more
      explanation.
    \end{tabular} \\
    \texttt{KR} & An object containing Kernel Regression estimates of the
    density of points, derived from the grouped data points. All
    fields correspond to \texttt{KDE} above.\\
    \hline
  \end{tabularx}
  \caption{Variables exported in the \texttt{r.mat} file.}
  \label{tab:matlab-export}
\end{table}



\section{Running a batch of reconstructions}
\label{manual:sec:runn-batch-reconstr}

The \textsc{Retistruct} library can be used to reconstruct a batch of
retinae which have been marked up. Suppose that the directory
\texttt{retinae} contains a directory tree in which there are data
directories containing the raw outline, data point and image files and
the saved markup files. In order to perform the reconstructions, we
create a new directory \texttt{retinae/reconstructions}, and run the
following sequence of commands in R:
\begin{verbatim}
R
> library(retistruct)
> retistruct.batch(tldir='retinae', outputdir='retinae/reconstructions')
\end{verbatim}
This command will go through the \texttt{retinae} directory, looking
for valid data directories. If it finds one, it sets about trying to
reconstruct the retina. As it reconstructing each retina, it writes to
log file in \texttt{retinae/reconstructions}. Once the reconstruction
is complete, it saves a number of plots in this directory in PDF
format. It also adds a line to a summary log file in
\texttt{retinae/reconstructions} called
\texttt{retistruct-batch.csv}. This file contains a number of columns:
\begin{description}
\item[\texttt{Dataset}] The directory of the data set
\item[\texttt{Return}] The return value from the process
\item[\texttt{Result}] A summary of the result, including if any
  errors were returned
\item[\texttt{E}] The total error  of the optimised reconstruction
\item[\texttt{El}] The error due to purely to the lengths of links in
  the optimised reconstruction
\item[\texttt{nflip}] The number of flipped triangles
\item[\texttt{EOD}] The distance of the Optic Disc from the inferred
  centre of the retina, in degrees. If the OD has not been marked up,
  this is \texttt{NA}.
\end{description}

To export the reconstruction data in a directory hierarchy in which
\texttt{retistruct.batch()} has been run, run the following sequence
of commands in R:
\begin{verbatim}
R
> library(retistruct)
> retistruct.batch.export.matlab(tldir='retinae')
\end{verbatim}


\appendix

\chapter{The IDT Data format}
\label{manual:sec:reading-data}

\begin{table}
  \begin{tabular}{ll}
    \hline
    \multicolumn{2}{c}{\textbf{FOR EACH BOUNDARY}} \\
    \hline
    \texttt{MAPNUM}   & id number of boundary \\  
    \texttt{MINLAT}   & min latitude      \\
    \texttt{MAXLAT}   & max latitude      \\
    \texttt{MINLON}   & min longitude     \\
    \texttt{MAXLON}   & max longitude     \\
    \texttt{LABLAT}   & latitude of label \\
    \texttt{LABLON}   & longitude of label\\
    \hline
    \multicolumn{2}{c}{\textbf{FOR EACH CELL}} \\
    \hline
    \texttt{XRED}     & $x$-coordinate if cell labelled red but not doubly\\
    \texttt{YRED}     & $y$-coordinate if cell labelled red but not doubly\\
    \texttt{XGREEN}   & $x$-coordinate if cell labelled green but not doubly\\
    \texttt{YGREEN}   & $y$-coordinate if cell labelled green but not doubly\\
    \texttt{XDOUBLE}  & $x$-coordinate if cell labelled doubly\\ 
    \texttt{YDOUBLE}  & $y$-coordinate if cell labelled doubly\\
    \texttt{XGRID}    & sample box cell is in \\
    \texttt{YGRID}    & sample box cell is in \\
    \texttt{PERIM}    & perimeter of cell \\
    \texttt{AREA}     & area of cell \\
    \hline
    \multicolumn{2}{c}{\textbf{ONE PER GRID BOX}} \\
    \hline
    \texttt{GRIDX}    & grid location of centre of sample box \\
    \texttt{GRIDY}    & grid location of centre of sample box \\
    \texttt{XGRIDCOO} & $x$-coordinate of centre of sample box \\
    \texttt{YGRIDCOO} & $y$-coordinate of centre of sample box \\
    \texttt{BOXSIZEX} & size (half width) of sample box in $x$-direction \\
    \texttt{BOXSIZEY} & size (half width) of sample box in $y$-direction \\
    \texttt{COMPLETE} & whether counting of sample has been completed\\
    \texttt{TOTALCEL} & total number of cells in this box\\
    \texttt{TOTALRED} & total number of red-only cells in this box\\
    \texttt{TOTALGRE} & total number of green-only cells in this box\\
    \texttt{TOTALDOU} & total number of double cells in this box\\
    \texttt{MEANPERI} & average perimeter of all cells \\
    \texttt{MEANAREA} & average area of all cells \\
  \end{tabular}
  \caption{Column headings of the \texttt{SYS.SYS} file.}
  \label{tab:data-format}
\end{table}

The data for each retina is stored in a separate directory. Within
each directory there are two files:
\begin{description}
\item[\texttt{SYS.SYS}] A table in SYSTAT format containing the
  coordinates of the red, green and doubly labelled cell bodies, and
  counts of labelled cell bodies within each grid box. The column
  headings shown in Table~\ref{tab:data-format}.  Each row of the
  table contains information only on a subset of the data, e.g.\ the
  coordinates of a red-labelled cell.
\item[\texttt{ALU.MAP}] A text file containing the coordinates of the
  map outline. The file comprises a number of sections, each starting
  with a single number, which is the number of lines to read in the
  next section. These lines have two numbers each, the $x$ and $y$
  coordinates of a vertex of the map outline.
\end{description}

% \section{Reading in and displaying data}
% \label{manual:sec:datafile-utils}

% All code is to be found in the \texttt{trunk/src} directory. The R
% program should be started from this directory in the following examples.

% To read in data use the functions in \texttt{datafile-utils.R}. Here
% is code to read in the data from a directory containing the
% \texttt{SYS.SYS} and \texttt{ALU.MAP} files, as detailed above:
% \begin{verbatim}
% source("datafile-utils.R")
% sys <- read.sys("/data/path/gm257-1-P8-C57BL6/")
% map <- read.map("/data/path/gm257-1-P8-C57BL6/")
% plot.sys.map(sys, map)
% \end{verbatim}

\begin{thebibliography}{99}
\bibitem{SterrattEtal2012} Sterratt DC, Lyngholm D, Willshaw DJ,
  Thompson ID (2013) Standard anatomical and visual space for the
  mouse retina: computational reconstruction and transformation of
  flattened retinae with the Retistruct package \newblock PLoS
  Comp. Biol. 9
\end{thebibliography}

\end{document}

%%% Local Variables: 
%%% TeX-PDF-mode: t
%%% End: 

% LocalWords:  MAPNUM MINLAT MAXLAT MINLON MAXLON LABLAT LABLON XRED YRED XGRID
% LocalWords:  XGREEN YGREEN XDOUBLE labeled YGRID PERIM GRIDX GRIDY XGRIDCOO
% LocalWords:  YGRIDCOO BOXSIZEX BOXSIZEY TOTALCEL TOTALRED TOTALGRE TOTALDOU
% LocalWords:  MEANPERI MEANAREA SYS SYSTAT ALU src datafile utils myheadings
% LocalWords:  Ubuntu retistruct Datapoints PDF csv Dataset nflip EOD YDOUBLE
% LocalWords:  Matlab cd matlab lX Dss Tss Sss IDT ImageJ png Analyze roi DV xs
% LocalWords:  Metadata metadata datapoints polararea DssMean KDE ys Xcode App
%  LocalWords:  Ctrl gui OpenOffice Calc Karcher DssHullarea flevels Xquartz KR
%  LocalWords:  maxs Lattitude gpa Sterratt Lyngholm Willshaw PLoS GTK retinae
% LocalWords:  alt isn checkboxes Markup markup toolbar min max von
%  LocalWords:  Csefalvay's workspace od ReconstructedOutline getDss
%  LocalWords:  ReconstructedDataset getSss Xenial Github datacounts
%  LocalWords:  Wulff Az El uncheck lll Azimuthal Conformal
