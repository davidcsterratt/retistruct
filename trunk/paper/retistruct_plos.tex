% Template for PLoS
% Version 1.0 January 2009
%
% To compile to pdf, run:
% latex plos.template
% bibtex plos.template
% latex plos.template
% latex plos.template
% dvipdf plos.template

\documentclass[10pt]{article}

% amsmath package, useful for mathematical formulas
\usepackage{amsmath}
% amssymb package, useful for mathematical symbols
\usepackage{amssymb}

% graphicx package, useful for including eps and pdf graphics
% include graphics with the command \includegraphics
\usepackage{graphicx}

% cite package, to clean up citations in the main text. Do not remove.
\usepackage{cite}

\usepackage{color} 

% Use doublespacing - comment out for single spacing
%\usepackage{setspace} 
%\doublespacing


% Text layout
\topmargin 0.0cm
\oddsidemargin 0.5cm
\evensidemargin 0.5cm
\textwidth 16cm 
\textheight 21cm

% Bold the 'Figure #' in the caption and separate it with a period
% Captions will be left justified
\usepackage[labelfont=bf,labelsep=period,justification=raggedright]{caption}

% Use the PLoS provided bibtex style
\bibliographystyle{plos2009}

% Remove brackets from numbering in List of References
\makeatletter
\renewcommand{\@biblabel}[1]{\quad#1.}
\makeatother


% Leave date blank
\date{}

\pagestyle{myheadings}

%% ** EDIT HERE **

\newcommand{\myshortjournaltitles}{}

%% ** EDIT HERE **
%% PLEASE INCLUDE ALL MACROS BELOW

\usepackage{amsmath}
\usepackage{amssymb,amsfonts,textcomp}
\newcommand\textstyletextbf[1]{\textbf{\textup{#1}}}
\newcommand\textstylereferenceref[1]{#1}
\newcommand\textstylecaptiontitle[1]{\textbf{\textup{#1}}}
\newcounter{Figure}
\renewcommand\theFigure{\arabic{Figure}}
\newcommand\normalsubformula[1]{\text{\mathversion{normal}$#1$}}
\usepackage{array}
\usepackage{supertabular}

\makeatletter
\newcommand\arraybslash{\let\\\@arraycr}
\makeatother
\newcounter{Text}
\renewcommand\theText{\arabic{Text}}

%% END MACROS SECTION

\begin{document}
\begin{flushleft}
\textbf{Standard anatomical and visual space for the mouse retina:
computational reconstruction and transformation of flattened
retinae\newline
}David C. Sterratt\textsuperscript{1*\#}, Daniel
Lyngholm\textsuperscript{2\#}, David Willshaw\textsuperscript{1}, Ian
D. Thompson\textsuperscript{2} \newline
\textbf{1} Institute for Adaptive and Neural Computation, School of
Informatics, University of Edinburgh, Edinburgh, Scotland, UK\newline
\textbf{2} MRC Centre for Developmental Neurobiology, King's College
London, London, UK\newline
*E-mail: david.c.sterratt@ed.ac.uk \newline
\textsuperscript{\#}These authors contributed equally to this work
\end{flushleft}

\section*{Abstract}
The concept of topographic mapping is central to the understanding of
the visual system at many levels, from the developmental to the
computational. It is important to be able to relate different 
coordinate systems, e.g. maps of the visual field and maps of the
retina. Retinal maps are frequently based on flat-mount preparations.
These use dissection and relaxing cuts to render the quasi-spherical
retina into a 2D preparation. The variable nature of relaxing cuts
and associated tears limits quantitative cross-animal comparisons.

We present an algorithm, ``Retistruct'', that reconstructs retinal
flat-mounts by mapping them into a standard, spherical retinal space.
This is achieved by: stitching the marked-up incisions of the
flat-mount outline; dividing the stitched outline into a mesh whose
vertices then are mapped onto a curtailed sphere; and finally moving
the vertices so as to minimise a physically-inspired deformation
energy function. Our validation studies indicate that the algorithm
can estimate the position of a point on the intact retina to within
8$^{\circ}$ of arc (3.6\% of nasotemporal axis).  The coordinates in
reconstructed retinae can be  transformed to visuotopic coordinates.

Retistruct is used to investigate the organisation of the mouse visual
system. We orient the retina relative to the nictitating membrane and
compare this to eye muscle insertions. To align the retinotopic and
visuotopic coordinate systems in the mouse, we utilised the geometry
of binocular vision. In standard retinal space, the composite
decussation line for the uncrossed retinal projection is located
64$^{\circ}$ away from the retinal pole.  Projecting anatomically defined
uncrossed retinal projections into visual space gives binocular
congruence if the optical axis of the mouse eye is oriented at
64$^{\circ}$ azimuth and 22$^{\circ}$ elevation, in concordance with previous
results. Moreover, using these coordinates, the dorsoventral boundary
for S-opsin expressing cones closely matches the horizontal meridian.

\section*{Author Summary}
There are a number of questions about the development and function of
the visual system that have been addressed by labelling cells in the
retina, then removing the retina and flattening it in order to
examine the labelled cells under the microscope. Some parts of the
retina that were originally together are now separated by the cuts
needed to flatten the retina. We have developed a computer program
that allows the flattened retina to be reassembled virtually onto a
sphere that represents a standard retinal space. This then allows us
to infer how retinal locations relate to positions in visual space.
We use this program to show conclusively that neurons activated by
blue light respond to parts of the visual field above the horizon.
Moreover, by injecting tracer into the visual areas in the left and
right hemispheres of the brain, we show which parts of the visual
field are binocular or monocular.

\section*{Introduction}
Understanding  how the retina maps onto different target regions, such
as the mammalian superior colliculus or lateral geniculate nucleus,
is a central feature of visual neuroscience \cite{DragerHubel1976,ColemanEtal2009}. However, there is considerable variation in the
descriptions of the mappings. Anatomical studies tend to focus on
\textit{retinotopic} coordinates, examining the mapping of dorsal
versus ventral (DV) retina and nasal versus temporal (NT) retina \cite{ReberEtal2004,RashidEtal2005}. Functional studies focus on
\textit{visuotopic} mappings: upper versus lower and central versus
peripheral visual field \cite{DragerHubel1976,DragerOlsen1980,HausteadEtal2008}.  The relation between retinotopic and
visuotopic maps is simplest when the latter is centred on the optical
axis of the eye but is more complicated for head-centred visuotopic
coordinate systems, especially in laterally-eyed animals.
Transformation between these coordinate systems is not intrinsically
problematic but does require knowledge of where the optic axis
projects in visual space. However, before reaching this stage there
is a more fundamental problem: reconstructing the retina.

Historically, the anatomical organization of the retina was examined
using serial sections, with the emphasis on example sections rather
than  reconstructions. The introduction of retinal flat-mounts, also
termed whole-mounts, \cite{Stone1981} was a major advance. Briefly,
orienting marks are made in the retina whilst in the eye-cup, the
retina is then dissected out and flattened with the help of a number
of relaxing cuts. The flat-mount facilitated quantitative
descriptions of the 2D distributions of different labels and markers
across the retina. However, the relaxing cuts, together with tears
that can occur during flattening, disturbs the retinal geometry
significantly, which not only complicates comparison across retinae,
but also can be problematic in interpreting results obtained from
individual flat-mounted retinae. For example, various measures are
used to quantify the regularity of mosaics of various cell types seen
in flat-mounted retinae \cite{WassleBoycott1991,RavenEtal2003},
but these are susceptible to the existence of boundaries \cite{Cook1996},
both at the rim of the retina and those introduced by the relaxing
cuts. In the study of topographic mapping, the locations of ganglion
cells labelled by retrograde tracers injected into different
locations in the target, the superior colliculus, have been compared
in retinal flat{}-mounts \cite{ReberEtal2004,RashidEtal2005}.
Foci of labelled cells can be separated, or even split, by relaxing
cuts (see Figure~1A), complicating quantitative analyses.

In the first part of the Results, we describe a method to infer where
points on a flat-mount retina would lie in a standard, intact retinal
space. The standard retina is approximated as a partial sphere, with
positions identified using spherical coordinates. Our results show
that the method can estimate the location of a point on a
flat-mounted retina to within 8$^{\circ}$ of arc of its original
location, or 3.6\% of the NT or DV axis. This has allowed us to
define a standard retinal space for the adult and for developing
mouse eye. In the second part of the Results we use the
reconstruction algorithm and the standard retinal space. Establishing
the orientation of retinal space for the mouse, whose retina contains
no intrinsic markers, requires experimental intervention. We show
that a mark based on the centre of the nictitating membrane is
reliable and  this mark can be related to the insertion of the rectus
eye muscles. Furthermore, we transform standard retinal space into
visuotopic space and use the geometry of binocular vision, together
with anatomical tracing, to address questions about the projection of
the optical axis of the mouse eye into visual space \cite{Drager1978,OommenStahl2008,ColemanEtal2009}. 

The reconstruction and transformation methods have been implemented as
an R package ``Retistruct''
(http://retistruct.r-forge.r-project.org/). Retistruct not only
facilitates comparison of differential retinal distributions across
animals but also allows analyses of distributions of labelled cells
in spherical coordinates, obviating the distortions associated with
2D flat-mounts. Finally, transforming retinal coordinates into visual
coordinates gives insights into the functional significance of
retinal cell distributions.  

\section*{Results}
\subsection*{The reconstruction algorithm}
In this section we give an overview of the reconstruction algorithm; a
more detailed description is contained in the Materials and Methods.
The starting point of the algorithm is the flattened retinal outline
(Figure 1A), which can include an image or labelled features. In
Figure 1A the outline includes a landmark (blue line), in this case
the optic disc, and the locations of retinal ganglion cells that have
been retrogradely labelled following discrete injections of red and
green fluorescent tracers into a  retino-recipient central nucleus
(the superior colliculus). One of the relaxing cuts has bisected the
labelled foci -- principally the red one. The first step is to mark
the nasal retinal pole (in Figure~1B, ``N''), which is defined by the
perimeter of the long cut towards the optic disc from the peripheral
fiducial mark based on the nictitating membrane (see Materials and
Methods). The locations and extents of incisions and tears in the
outline are also marked up (cuts 1-4 in Figure~1B). Because the
retina in the eye-cup is more than hemispherical, the angle of the
retinal margin (rim angle) measured from the pole of the retina is
then supplied (Table 1 and see later section below). The outline is
then divided into a mesh containing at least 500 triangles of roughly
equal size, and the incisions and tears are stitched together
(Figure~1C). This mesh is then projected onto a spherical surface,
with all the points on the retinal margin lying on the circle defined
by the rim angle (Figure~1Di). Each edge in the mesh is treated as a
spring whose natural length is the length of the corresponding edge
in the flat mesh. It is not possible to make this initial mapping
onto the spherical surface optimal, so the springs are either
compressed (blue), expanded (red) or retain their natural length
(green). In the next step the springs are allowed to relax so as to
minimise the total potential energy contained in all the springs,
leading to the refined spherical mesh shown in Figure~1Ei. 

The locations of mesh points in the flat-mount and their corresponding
locations on the sphere define the relation between any point in the
flat-mount and a standard spherical space. This relation  is used to
map the locations of data points and landmarks in the flat retina
onto the standard retina. These can be visualised interactively on a
3D rendering of a sphere (see Figure~1D), or represented using a map
projection such as the azimuthal equidistant projection centred on
the retinal pole (Figure~1Fi). In all plots of retinal space we use
the colatitude and longitude coordinate system, where colatitude is
like latitude measured on the globe, except that zero is the pole
rather than the equator. Map projections \cite{UGS2006} such as that in
Figure 1Fi are very useful for rendering retinal label into a
standard retinal space.  An alternative representation is shown in
Figure~1Dii,Eii,Fii, where lines of latitude and longitude on the
standard spherical retina have been projected onto the flat
structure. The jump between Figure~1Dii and 1Eii illustrates the
improvement in the appearance of the mapping achieved after
minimising the energy. 

In order to analyse data points on the standard retina, we used
spherical statistics \cite{FisherEtal1987}. The mean locations of
groups of data points on the sphere (diamonds in Figure 1Fi) are
computed using the Karcher mean and we used density estimation to
produces contour plots (see Materials and Methods). 

\subsection*{Performance of the algorithm}
To assess the amount of residual deformation at the end of the energy
minimisation procedure, we plotted the length of each edge $i$
in the spherical mesh  $l_i$ versus the length of the corresponding
edge in the flat mesh  $L_i$ (Figure 2A), using the same  colour
scale as in Figure 1D,E. A measure of the overall deformation of
reconstruction is:
\begin{equation}
e_{\text L}=\sqrt{\frac 1{2|\mathcal{C}|\overline L}\sum _{i\in\mathcal{C}}\frac{(l_i-L_i)^2}{L_i}}
\end{equation}
where the summation is over $\mathcal{C}$, the set of edges, the mean
length of an edge in the flat mesh is  $\overline L$, and the number
of edges is {\textbar}$\mathcal{C}${\textbar}. Physically, this measure
is the square root of the elastic energy contained in the notional
springs. It is constructed so as to be of a similar order to the mean
fractional deformation.

We used our algorithm on 297 flat-mounted retinae, 288 of which could
be reconstructed successfully, 7 of which failed due to as-yet
unresolved software bugs and 2 of which were rejected because of
unsatisfactory reconstructions (see below). Figure~2A shows the
reconstruction with the lowest deformation measure
$e_\mathrm{L}=0.038$ and Figure~2E the example with a
high value of $e_\mathrm{L}=0.118$. The arrangement of
the grid lines in the example with lower deformation looks
qualitatively smoother and more even than in the example with higher
deformation measure (Figure~2C,G). The strain plot for the retina
with the lower deformation (Figure~2B) indicates that almost all
edges are unstressed, whereas in the retina with higher deformation
(Figure~2F) there are many more compressed and expanded edges. It can
be seen that the retina in Figure~2E{}-H has a much less distinct
margin than in Figure~2A-D. This makes it harder for the algorithm to
create an even mapping, as local roughness in the rim forces
significant deformation of the surrounding virtual tissue when
morphed onto the sphere. 

Thus the deformation measure $e_\mathrm{L}$ gives some
indication of the apparent quality of the reconstruction.  A value
greater than 0.2 indicates a problem with the stitching part of the
algorithm; the 2 such reconstructions were rejected and are not
included in the 288 successful reconstructions. Noticeably bad
reconstructions tend to have
$e_\mathrm{L}>0.1$. The mean deformation
measure was 0.071, with a median value of 0.070 and with 27 out of
288 retinae had a error measure exceeding 0.1 (Figure~3A). The larger
deformations tend to come from younger animals (Figure~3B),
reflecting the difficulty of dissecting retinae out of these animals
cleanly due to the more delicate nature of younger tissue. 

\subsection*{Determination of the rim angle}
To determine the rim angle of mouse eyes at varying stages of
development we measured the distance $d_\mathrm{e}$ from
the back of the eye to the front of the cornea and the distance
$d_\mathrm{r}$ from the back of the eye to the edge of
the retina (Figure~3C). We then computed the colatitude (the angle
measured from the retinal pole)   $\varphi _0$ of the rim using the
formula:
\begin{equation}
\varphi _0=\arccos (1-2d_{\text r}/d_{\text e})
\end{equation}
The measurements and derived latitudes for animals of various ages are
shown in Table 1. 

An alternative approach to setting the rim angle is to infer, for each
individual retina, the rim angle that minimises the deformation. This
was done by repeating the minimisation for rim angles at 1$^{\circ}$
intervals within a range  $[-20^{\circ },+5^{\circ }]$ of the rim
angle determined by measurement as above. A comparison of the
measured and inferred rim angles  is shown in Figure~3D. It can be
seen that inferred rim angle is usually less than that obtained from
measurements of standard retinae. Figure~3E shows a comparison of the
 reconstruction error obtained using the measured and inferred rim
angles. The maximum decrease in the reconstruction error is 19.1\%,
with the mean improvement being 7.2\%. We concluded that this
improvement was not sufficiently great to add automatic refinement of
the rim angle to the algorithm.

\subsection*{Estimate of errors of the reconstruction algorithm from
optic disc location}
The deformation measure gives an indication of how easy it is to morph
any particular flattened retina onto a partial sphere, but it does
not indicate the error involved in the reconstruction, i.e. the
difference between the inferred position on the spherical retina and
the original position on the spherical retina. The ideal method for
estimating the error would be to flatten a retina marked in known
locations, and then compare the inferred with the known locations.
However, this proved to be technically very difficult and so we tried
another method of estimating the error, that uses the inferred
locations of the optic discs across a number of retinae. In mice, the
optic disc is located ``rather precisely in the geometric center of
the retina'' \cite{DragerOlsen1981} though this has not, as far as
we are aware, been measured. We marked the optic disc in 72 
flat-mounted adult retinae, and the distribution of the centres of
the inferred locations of these optic discs is shown in Figure~4A,B.
The mean colatitude and longitude of these optic discs is (3.7$^{\circ}$,
95.4$^{\circ}$) with a standard deviation of 7.4$^{\circ}$. The mean is
therefore 3.7$^{\circ}$ away from the geometrical centre of the retina,
in good agreement with the qualitative observation that the optic
disc is at the geometric centre of the retina. Under the,
questionable, assumption that none of the variability is biological,
this suggests that an upper bound on the accuracy of the
reconstruction algorithm is 7.4$^{\circ}$. There is a significant
relationship between the deformation error
$e_\mathrm{L}$ and the inferred distance of the optic
disc from the mean optic disc location (Figure~4C). If
reconstructions requiring accuracy to less than 7.4$^{\circ}$, this could
be achieved by increasing the stringency of measure
$e_\mathrm{L}$ values used to reject reconstructions.
Rounding up this error gives a value of 8$^{\circ}$, which is 3.6\% of
the 220$^{\circ}$ along the nasotemporal axis in the smallest eye
considered.

\subsection*{Distributions of ipsilateral and contralateral retinal
projections }
We now describe the first application of the reconstruction algorithm.
The relationship between the projections from the mouse retina to the
ipsilateral and contralateral dLGN has been studied in retina
flat-mounts following injection of retrograde tracer into the dLGN \cite{ColemanEtal2009}. Reconstructing the retinae of individual
animals that have had retrograde tracer injected into primary visual
areas enables comparisons of projection patterns across animals
independent of distortions introduced by retinal dissection.
Moreover, having a standard retinal space means that data from
multiple animals can be used to create aggregate topographic maps. 

The reconstruction method has enabled the quantification of the
binocular projection from the retina to the dLGN across multiple
animals. To label the projection, the dLGN was injected either with
Fluoro-Ruby or Fluoro-Emerald. Further, in some animals, the
injections were bilateral (see Figure~5A and Materials and Methods
for details). The Retistruct program was used to reconstruct the
retinae. The plots in Figures~5B-D show the label in \textit{one}
retina from an animal that had received \textit{bilateral} injections
into the dLGN (Figure~5A). These plots were done using an in-house
camera-lucida set-up, sampling the entire ventrotemporal crescent for
the ipsilateral retina and one in nine 150\,{\textmu}m square boxes
for the contralateral label. The ipsilateral projection (Figure~5C)
is restricted to the ventrotemporal crescent whereas label from the
contralateral projection (Figure~5D) is distributed widely. The
nature of the overlap between the uncrossed and crossed projections
is evident in Figure~5B. Having reconstructed the retinae into a
standard space, we quantified the projection patterns by deriving
kernel density estimates (KDEs) of the underlying probability of data
points appearing at any point in the retina and represented these
estimates using contours that exclude 5\%, 25\%, 50\% and 95\% of the
points (Figure 5C). In the case of the contralateral label, the data
consisted of cell counts within defined boxes on the flattened
retina. Here we used kernel regression (KR) as the source for the
contouring algorithm (Figure~5D; see Materials and Methods for
details). The Karcher mean of the data points is represented by the
red and green diamonds in either plot and the peak density for the
kernel is represented by the blue diamond. It is worth noting that
these two measures often give different locations, as would be
expected from skewed distributions..  

To measure the extent of the ipsilateral projection, we made a
composite plot of data from 7 different animals (Figure~5E), which
shows that the average ipsilateral projection occupies a crescent in
ventrotemporal retina. The decussation line for the aggregate
ipsilateral population is 64.1{\textpm}1.6$^{\circ}$ from the retinal
pole, which in these 7 animals is very close to the optic disc. The
distance from the optic disc to the decussation line is
63.4{\textpm}1.3$^{\circ}$ (Figure~5F). The ipsilateral crescent spans an
average of 134.1{\textpm}1.5$^{\circ}$ of the rim extending from
22.1{\textpm}1.3$^{\circ}$ beyond the temporal pole  to
22.0{\textpm}1.5$^{\circ}$ beyond the ventral pole (Figure~5G).

\subsection*{Transformation to visuotopic coordinates}
The geometry of binocular vision implies that the ipsilateral
decussation line should correspond to the vertical meridian in the
mouse's visual field \cite{DragerOlsen1980,ColemanEtal2009}. In
order to investigate this prediction, we sought to map the retina
onto visual space. The mapping of visual space onto the retina is
determined by the orientation of the optic axis and the optics of the
eye. We assume that the optic axis corresponds to the retinal pole of
the spherical retinal coordinate system. Thus the optic axis is close
to but not coincident with the optic disc. The location of the optic
\textit{disc} has been estimated to be projected to a point 60$^{\circ}$
lateral to the vertical and 35$^{\circ}$ above the horizontal meridian in
anaesthetised mice \cite{Drager1978}. Alternatively, the optic
\textit{axis} has been reported to be 64$^{\circ}$ lateral to the
vertical and 22$^{\circ}$ above the horizontal meridian in ambulatory
mice \cite{OommenStahl2008}. In anaesthetised mice,  Dr\"ager and
Hubel noted that the eyes were always diverted outwards \cite{DragerHubel1976}.

In principle, the deviation of a ray by the eye can be estimated by
means of a schematic model of the mouse eye \cite{RemtullaHallett1985,SchmuckerSchaeffel2004}. We investigated using one such
model \cite{SchmuckerSchaeffel2004} to determine the deviation of
paraxial rays. The schematic eye model is not, however, constructed
to account for wide-field rays, so we decided to approximate the
effect of the optics of the eye by making the deviation of a rays
passing through the posterior nodal point of the eye, which is
approximately the centre of the eye, proportional to the ray's angle
of incidence. The constant of proportionality is such that rays at
90$^{\circ}$ to the eye will be projected onto the edge of the retina,
regardless of the retina's rim angle (see Figure~6B).  The mapping of
the eye onto visual space is effected by a coordinate transformation
in which first the approximate optics are used to project points on
the retina through the centre of the eye onto a notional large
concentric sphere about the eye representing visual space. Then the
locations of the points on this ``celestial'' sphere are measured in
terms of elevation, the angle above the horizontal and azimuth, the
angle made between the point's meridian plane and the zero meridian
plane, i.e. the vertical plane containing the long axis of the mouse.
By convention \cite{DragerOlsen1980,ColemanEtal2009,Drager1978}, projections of visual space are presented as though the mouse
is sitting facing the observer, so that the azimuth angle is positive
in the left visual field. 

Using the above conventions to test whether the position of the
ipsilateral decussation line corresponds to the vertical meridian in
visuotopic space, we have transformed the retinotopic location of
ipsilateral retinal ganglion cells, following bilateral injections,
into head-centred visuotopic space \cite{BishopEtal1962}. To minimise
between-animal variability, we used bilateral injections into the
dLGN: injecting Fluoro-ruby on one side and Fluoro-emerald on the
other side (see Figure~5A). Figure~6A illustrates the distribution of
retrogradely-labelled ganglion cells in the ipsilateral (upper plots)
and contralateral (lower plots) retinae. For this Figure, we have
abandoned the standard representation of the nasal retina to the
right in order to emphasise the mirror-symmetry of the projections.
Injection of Fluoro-emerald into the right dLGN leads to label
restricted to the ventrotemporal crescent in the right retina but
widespread labelling in the left retina; a complementary pattern is
seen for an injection of Fluoro-ruby into the left dLGN. When the
retinal distribution of the ipsilateral ventrotemporal crescent
neurons is transformed into visuotopic space using the optic axis
coordinates of 64$^{\circ}$ azimuth and 22$^{\circ}$ elevation \cite{OommenStahl2008}, and plotted in an orthographic projection of the central
visual field, the decussation line lines up with the vertical
meridian. These plots have been rotated with 50$^{\circ}$ elevation to
include the upper part of the visual field beyond 180$^{\circ}$
(Figure~6C). If, in contrast, the ipsilateral projection is projected
using the optic disc coordinates of 60$^{\circ}$ azimuth and 35$^{\circ}$
elevation \cite{Drager1978}, there is an evident visual mismatch between
the two decussation lines (Figure~6D). To examine the visuotopic
extent of the contralateral projection in the two eyes, we displayed
the data in a sinusoidal map projection to include the full visual
field through both eyes (Figure~6E). This demonstrates that the
decussation pattern in the contralateral projection is not as sharp
as that in in the ipsilateral projection. Intriguingly, it appears
that inferior-central field, a region that would be shadowed by the
nose, is also under-represented. 

\subsection*{Eye muscles and S-opsin in retinal and visuotopic
coordinates}
To examine the orientation of the eye, the locations of the insertions
of superior, lateral and inferior rectus into the globe of the eye
were marked onto the retina (Figure~7A; see Materials and Methods for
procedure). The nasal pole of the retinae is determined with
reference to the nictitating membrane. The retinae were reconstructed
and plotted in an azimuthal equidistant polar projection and the
vectors connecting the insertion points and the optic disc were
plotted (Figure~7B). Once in a standard space, the muscle insertion
points ($n=33$) from all the retinae ($n=17$) were
plotted in the same plot and the vectors connecting the Karcher mean
of each muscle insertion and the Karcher mean for the optic disc
location were plotted (Figure~7C) with the result that the mean angle
of the lateral rectus vector, at 184.9{\textpm}3.6$^{\circ}$, is directly
opposite the nasal cut. The superior rectus vector is at
91.3{\textpm}5.9$^{\circ}$ and the inferior rectus is at
284.2{\textpm}4.1$^{\circ}$ (where nasal is 0$^{\circ}$). It is noticeable
that there is considerable variability in the locations of the muscle
insertions, certainly when compared to the variability of the optic
discs. A considerable contributory factor in this is the large extent
of the muscle and the relative difficulty in determining the centre
of the muscle.

The location of the optic axis at 64$^{\circ}$ azimuth and 22$^{\circ}$
elevation \cite{OommenStahl2008}  determines the location of the
vertical and horizontal meridians on the eye. The location of the
ipsilateral decussation line confirms this azimuthal value
(Figures~5E \& 6C). As the mouse retina has no pronounced horizontal
streak, we looked at the distribution of short wavelength opsin
(S-opsin) in the retina. Haverkamp et al. \cite{HaverkampEtal2005}
describe a very distinct distribution pattern in the retina, with
high density ventral, low density dorsal and an abrupt transition and
suggested that the transition coincided with the horizontal meridian.
Figure~8A shows immuno-staining for S-opsin from dorsal, central and
ventral retina. The density difference between dorsal and ventral is
marked and the transition is abrupt. Figure~8B \&~8C illustrate the
transformation of the S-opsin distribution from retinal flat-mounts
to standard retinal space to an orthographic representation of
visuotopic space centred on the optic axis. In these plots, the
transition is 3.3{\textpm}0.3$^{\circ}$ above the horizontal meridian at
the level of the optic axis and is tilted by 13.8{\textpm}3.6$^{\circ}$,
so that transition is 7.8{\textpm}2.0$^{\circ}$ above the horizontal
meridian in central visual field and 6.0{\textpm}1.6$^{\circ}$ below the
horizontal meridian peripherally (Figure~8D). The label from both
left and right retinae was also plotted in a sinusoidal projection
(Figure~8D) to illustrate that the rotation seen in the orthogonal
plots for each eye is symmetric along the vertical meridian of the
entire visual field (Figure~8E).

\section*{Discussion}
\subsection*{Assessment of the method}
As far as we are aware, a computational algorithm for reconstructing
retinal flat-mounts into standard space and then transforming these
retinal coordinates into visuotopic coordinates has never been
described before. Our validation studies suggest that the method is
able to estimate the original location of a point on a flattened
retina to within 8$^{\circ}$ of arc, which is equivalent to 3.6\% of the
nasotemporal axis.  The ability to reconstruct retinae allows for the
analysis of populations of labels that would otherwise be split
across the incisions or tears of the flat-mount preparation. It also
allows retinae from different eyes to be mapped onto a standard
retina, enabling cross-eye comparisons to be made. The ability to map
images into standard retinal space is a useful method of
visualisation, as are the various map projections from standard
retinal space. As part of the software package that implements the
algorithm, we have provided routines to analyse point data on the
sphere. 

It is important to understand the limitations of the method and how
they might be mitigated. Firstly, the algorithm can only be as good
as the experimental error of dissection and fixing fed into it. 
There is a great deal of variability in dissected retinae: in some it
is clear where the vertices of incisions and tears lie and in others
is much less so; thus error may be added when the incisions and tears
are marked up. The deformation measure gives an indication of the
quality of a reconstruction and was higher (worse) for younger
retinae (Figure~3B), which are more delicate and harder to dissect
cleanly.  A high deformation measure indicates that a reconstruction
should be treated with caution. In retinae where it less clear where
the tears and incisions are, trying an alternative marking up can
reduce the deformation.

Secondly, the physical model used by the algorithm contains
simplifying assumptions. The retinal tissue is modelled as a mesh of
masses connected by springs. This basic model is used in modelling
deformable surfaces in computer animation and clothes design \cite{FanEtal1998,McCartneyEtal1999,WangEtal2002} and has been used
in models of soft tissue \cite{SkrinjarDuncan1999}. This model
should ensure that neighbourhood relations are preserved -- nodes
close in the 2D structure should still be close in the 3D structure
-- and incorporates the notion of an elastic material. However, it
does not incorporate an explicit representation of tensile and shear
forces. 

A more physically-principled method of modelling deformation of
objects is offered by the finite element method (FEM), in which the
stress and strain of each triangular element in the mesh is derived
as a function of the deformation of the points of the element \cite{ZienkiewiczTaylor2000}. The stress and strain matrices
represent shearing within the material, and materials with varying
degrees of compressibility (as quantified by Poisson's ratio) can be
modelled. The FEM is used widely to describe properties of soft
tissue in simulations of surgery \cite{CarterEtal2005}. The FEM could
be applied fairly straightforwardly to the forward problem of how a
retina with known incisions deforms during flattening. However, it is
not possible to apply the FEM directly to the problem of mapping the
flattened retina onto the intact retina. This would require knowledge
of the stresses and strains in the flattened retina, whereas the only
information available is the outline of the retina and the location
of the optic disc.

Finally, and less crucially, the meshing algorithm is sensitive to
neighbouring points on the outline being very close together. This
leads to a large number of very small triangles being constructed,
which can lead to a tangle of many flipped triangles in the energy
minimisation phase and, ultimately, small areas of high strain in the
reconstructed outline. Suitable prepossessing of outlines to remove
very short edges can eliminate this problem. 

\subsection*{Possible extensions}
In terms of the reconstruction algorithm, it might be possible to
automate the mark-up procedure. This would be straightforward for
cleanly dissected retinae, but would be difficult for retinae with
irregular incisions and tears. Furthermore marking up takes at most a
few minutes using the interface supplied with the software. The other
piece of information required by the algorithm is  the rim angle,
which is currently determined by the age of the tissue (Table~1). It
would be possible to infer this automatically (Figure~3D) but we
decided that the extra complexity did not warrant doing this.

It would be possible to add extra analysis routines to the program.
When examining retinae with mosaic labeling \cite{HubermanEtal2008},
reconstructing the retina into its original spherical coordinates
would make it possible to determine more accurately the relative
distances between cells between peripheral and central retina. To
implement mosaic analysis would require computation of a Voronoi
tessellation on the sphere, which could be implemented by doing the
Voronoi tessellation on a conformal (Wulff) projection \cite{NaEtal2002}. 

Further visualisation methods could also be added. Using equations
similar to those in the literature \cite{BishopEtal1962}, locations on
the retina could be projected onto a screen placed orthogonal to the
optic axis, thus enabling a direct translation of visual stimulus
space onto retinal coordinates. It would also be possible to map
visual space onto retinal coordinates, by inverting the operations to
map the retina onto visual space. 

\subsection*{Biological insights gained from retinal
reconstructions and transformations}
The power of the Retistruct algorithm is that it allows information
from retinal flat-mounts to be reconstructed into spherical
coordinates. In generating the reconstructions, the algorithm
provides information about the accuracy of the reconstruction, so
that retinae damaged during dissection can be eliminated from the
sample on quantitative grounds. Retistruct allows features present on
the retina to be represented in standard retinal space, once the
orientation of the retina is fixed. It could be applied to flat-mount
preparations of retinae from any vertebrate species, provided the
globe of the eye is approximately spherical. Importantly,
reconstructing retinae provides a standardised coordinate system,
which enables the comparison of retinal features between multiple
animals of different ages. Plotting features such as photoreceptor
density and distribution of retinal ganglion cell subtypes in a
standard framework means that more rigorous analyses of the variance
between samples and across groups can be performed. Although this
paper is focussed on the mouse, transformations from retinal
coordinates into visuotopic coordinates would allow comparison of
retinal features across species. 

Standard retinal space permits the comparison of retinal features
across many animals, including retinae from animals of different
ages, with the caveat that there are subtle variations in retinal
space with age. The neonatal retina is much more spherical in extent
than the adult retina (see Table~1). We have not explored the
topological consequences of this change in retinal geometry for the
mapping of the retina onto its target nuclei, which are also changing
in shape in the postnatal period. Additionally, it would be possible
to generalise the method to deal with a retina modelled by any shape
with axial symmetry. This would just require a different energy
function to be used in the minimisation procedure. This might lead to
more accurate reconstructions, though would depend on knowing the
geometry of the intact retina more accurately than at present.

In defining standard retinal space, we fixed the nasal pole with
reference to the centre of the nictitating membrane. It is simple to
make a cut with a microknife into the cornea opposite this point
before extracting the eye at all ages from birth to adulthood. This
cut is then used to mark up a fixed nasal pole. When the insertions
of the eye muscle rendered into standard retinal space are plotted
with reference to this nasal pole, the nasal pole aligns with the
insertion of the lateral rectus muscle (see Figure~7D). Moreover, we
found that the accuracy in determination of the eye muscle insertions
was not great even in the adult (see Figure~7C). We believe that this
approach is a convenient and accurate method for determining the
nasal retinal pole. Having standardised retinal extent and
orientation, we were able to examine the distribution of retinal
features across animals.

The only feature obvious in an untreated retinal flat-mount of the
mouse is the optic disc. We find that the average location of the
optic disc in standard space is at (3.7$^{\circ}$, 95.4$^{\circ}$), which is
only 3.7$^{\circ}$ from the geometric centre of the retina. The slight
discrepancy would be difficult to establish from flat-mounts alone.
In terms of experimentally labelled retinal features, we have
focussed on the uncrossed retinal projection and the distribution of
S-opsin. It has been known, since Dr\"ager and Olsen \cite{DragerOlsen1980} that the majority of retinal ganglion cells projecting
ipsilaterally were located in the ventrotemporal crescent. In
flat-mounts, it is possible to count the numbers of cells and obtain
estimates of density distributions. Because we have been able to
combine projections across animal in standard space (see Figure~5E),
we can say that the crescent extends a little into dorsal and into
nasal retina (by about 22$^{\circ}$ in each case). Furthermore, the start
of the crescent is 64.1$^{\circ}$ from the retinal pole, or 63.4$^{\circ}$
from the optic disc. When considered in standard retinal space, the
distribution of the S-opsin is distinctive -- the dense staining
mostly confined to ventral retina with extensions into dorsal retina
over most of the nasotemporal axis. 

In fact, interpretation of the retinal distributions of both the
uncrossed ganglion cells and the S-opsin is best considered in
visuotopic space. The geometry of binocular vision means that the
central edge of the ventrotemporal crescent (the uncrossed
decussation line) should correspond to the vertical meridian of the
visual field \cite{DragerOlsen1980,ColemanEtal2009}.
Transformation from retinal polar coordinates to visuotopic
coordinates is relatively straightforward if it is known where the
mouse eye is looking in visual space. Dr\"ager \cite{Drager1978} reported
that the projection of the optic disc into visual space in a
head-fixed, anaesthetised mouse was 60$^{\circ}$ away from the vertical
meridian and 35$^{\circ}$ above the horizontal meridian.  However, Oommen
and Stahl \cite{OommenStahl2008} concluded from their imaging of the
eye of ambulatory mice that the optical axis was at 64$^{\circ}$ azimuth
and 22$^{\circ}$ elevation. In head-fixed but unanaesthetised mice,
compensatory eye movements kept the optical axis constant over a
large range of pitch. If compensatory eye-movements do not occur
under anaesthesia, then retinal projections will depend on head
inclination as Dr\"ager and Hubel \cite{DragerHubel1976} noted.
Although, the differences appear to be small, when we plotted the
locations in visual space of the uncrossed projections from both
eyes, the two uncrossed decussation lines lined up along the vertical
meridian with the optic axis location described by Oommen and Stahl \cite{OommenStahl2008} but not with the optic disc location of
Dr\"ager \cite{Drager1978}. Further, the distance from the retinal pole to
the edge of the decussation line in the grouped data was also
64$^{\circ}$. Given these observations and our group data for optic disc
location (Figure~5E), this indicates a revised optic disc projection
of 66$^{\circ}$ azimuth and 25$^{\circ}$ elevation. 

Having fixed the location of the eye in visual space, our data on the
uncrossed projection predict that the width of binocular visual field
is 52$^{\circ}$ at its greatest width, slightly larger than the
30-40$^{\circ}$ of Dr\"ager and Hubel \cite{DragerHubel1976}, but within
the range of 50-60$^{\circ}$ estimated by Rice et al. \cite{RiceEtal1995}
and Coleman et al. \cite{ColemanEtal2009}. The binocular field starts
a few degrees below the horizon and continues well behind the
animal's head -- very like the situation in the rabbit \cite{Hughes1971}.
Finally, with this location, the change in S-opsin staining nicely
coincides with the horizontal meridian, at least for central visual
field, as predicted by Sz\'el et al. \cite{SzelEtal1992}. It is worth
noting that S-opsin is mostly co-expressed with medium wavelength
opsin (M-opsin) \cite{HaverkampEtal2005,AppleburyEtal2000}, and
that this would make these cones in upper visual field respond to a
broader spectrum, which may have implications for the ability to
detect a larger range of objects above the mouse. The precise
distribution of S-opsin is not uniformly agreed upon \cite{HaverkampEtal2005,AppleburyEtal2000,SzelEtal1992}. However, when
plotting the distributions from Sz\'el et al \cite{SzelEtal1992} to
visuotopic space using Retistruct (data not shown), we get a very
similar distribution to that seen in Figure~8B-C. 

In conclusion, the Retistruct algorithm uses a stitching algorithm,
mesh-generation and energy minimisation to reconstruct retinal
flat-mounts into standard retinal space reliably. The algorithm
generates deformation measures that indicate the quality of the
reconstruction. For an individual retina, the distribution of retinal
features can be described much more accurately in spherical
coordinates, without the complications associated with flattening.
Standard retinal space allows comparison of labelling distributions
across animals. Finally, mapping retinal distributions onto visual
space gives insights into their functional significance.

\subsection*{}
\section*{Materials and Methods}
\subsection*{Animals}
All animals used in this study are on a C57BL/6J background (Charles
River, UK). All procedures were performed in accordance with the
European Communities Council Directive of 24 November 1986
(86/609/EEC) under the Animals (Scientific Procedures) Act 1986. For
surgery, animals were given an injection of Vetergesic (0.1\,mg/kg)
and anaesthetised either with Isoflurane (induction at 4\% and
maintenance at 2\%) or ketamine/xylazine (3\,ml/kg of stock solution:
33\,mg/ml, Ketamine and 3\,mg/ml, Xylazine). Animals were then secured
with ear-bars in a custom-made frame and maintained at 37$^{\circ}$C
throughout surgery. For injections into the dLGN, a skull flap was
drilled and either unilateral or bilateral cortical aspiration was
performed to reveal the thalamus. Subsequently, either
fluorescein-conjugated dextran (Fluoro-Emerald) or
tetramethylrhodamine-conjugated dextran (Fluoro-Ruby) (5\%~w/v;
10,000MW; D1820, D1817; Invitrogen) was pressure-injected through
glass pipettes with a 120\,{\textmu}m internal diameter and tip
diameters 30{}--100\,{\textmu}m. Multiple injections were made into
the thalamus; total volumes into one thalamus ranged between 50\,nl
and 250\,nl.  For injections into the superior colliculus, a skull
flap was similarly drilled and unilateral cortical aspiration was
performed to reveal the entire superior colliculus. Subsequently,
5{}--10\,nl of Red and Green latex microspheres (RetroBeads;
Lumafluor) were pressure-injected into the upper 200-300\,{\textmu}m
of the superior colliculus using similar pipettes as above, but with
30{}--50\,{\textmu}m tip diameters. Following surgery, the cavity from
aspiration was filled with absorbable gelatine sponge, the skull-flap
replaced and the skin sutured. Animals were allowed to recover and
left for 24--48 hours after which they were terminally anaesthetised
by intraperitoneal injection of pentobarbital sodium (300\,mg/kg) and
perfused transcardially with phosphate buffered saline at 4$^{\circ}$C
followed by paraformaldehyde (4\% w/v) at 4$^{\circ}$C.

\subsection*{Retinal dissections}
With the eye still in the head, a microknife cut was made in nasal
cornea at the level of the middle of the nictitating membrane.  This
allowed a subsequent orienting cut to be made in nasal retina. The
eye was removed and dissected using a Sylgard-filled Petri dish and
insect-pins. The front of the eye was removed just anterior to the
corneo-scleral junction and the lens removed. Fine scissors were used
to make a long cut through the sclera, choroid and retina from the
nasal orienting cut towards the optic disc. Shorter relaxing cuts
were made at approximately 90\textsuperscript{o} or
120\textsuperscript{o}  intervals, depending on the age of the
animal. The retina was freed from the choroid and cut free at the
optic disc. It was then transferred to a Poly-L-Lysine coated
microscope slide, mounted ganglion cell layer up and cover-slipped
using Fluoromount (Sigma). Retinae were imaged using an Axiophot2
microscope (Zeiss). Subsequent image processing and mark-up of the
outline was done using ImageJ (NIH, USA). Retinal ganglion cell
locations in retinae used for Figure~5, however, were plotted using
an in-house camera-lucida set-up. 

To investigate the orientation of the eye, the insertions of the
superior, inferior and lateral rectus muscles were marked onto the
retina by passing a needle coated with Fluoro-Emerald or Fluoro-Ruby
(5\% v/w) through the sclera into the retina at the centre of the
muscle-insertion prior to dissecting the retina. The staining for
shortwave opsin was done using a polyclonal antibody against S-opsin
(1:200; AB5407; Millipore) with a Fluorescein-conjugated secondary
antibody (1:1000; AP123F; Millipore).

\subsection*{Reconstruction algorithm}
The reconstruction algorithm described here has been implemented in R \cite{R2011} and can be downloaded as an R package from
http://retistruct.r-forge.r-project.org/. It has been tested on
GNU/Linux, but should also work in MacOS and Windows. The user
manual, available from the same site, contains details of the two
main data formats Reistruct can process.  These are either in the
form of coordinates of data points and retinal outline from an
in-house camera-lucida setup, or in the form of bitmap images with an
outline marked up in ImageJ ROI format (NIH, USA).  In this paper,
Retinae were imaged using an Axiophot2 microscope (Zeiss) and
processed to find cell locations using imageJ. Retinal ganglion cell
locations in retinae used for Figures~1,~2 and~5, however, were
plotted using an in-house camera-lucida setup.

The steps of the reconstruction algorithm are illustrated in Figure 1.
The raw data (Figure 1A) consists of the sequence of points making up
the outline, sets of data points and sequences of connected points
defining landmarks, such as the optic disc. The reconstruction
process then proceeds as follows: 

\begin{enumerate}
\item The location of one of the nasal position and all the incisions
and tears in the outline are marked up by an expert (Figure~1B) using
the custom-built graphical user interface. Each tear is defined by a
central point, referred to as the apex, and two outer points,
referred to as vertices. Tears within tears or incisions can be
marked up. 
\item The retinal outline is triangulated using the conforming
Delaunay triangulation algorithm provided by the Triangle package \cite{Shewchuk1996} such that there are at least 500 triangles in the
outline (grey lines in Figure 1C). 
\item The tears and incisions are stitched automatically (cyan lines
in Figure 1D). To do this, firstly the length of each side of every
tear is computed. The fractional distance of each point in each tear
is then defined as the distance along the tear of that point from the
apex divided by the total length of that side of the tear. For each
point on one side of a tear a corresponding point is inserted at the
same fractional distance along the opposing side. Tears within tears
are dealt with by excluding the child tear when computing the
fractional distance. At the end of the procedure there is a set of
correspondences between two or, in the case of the vertices of a
child tear, three points. 
\item There is then an extra round of triangulation to incorporate the
points that have been inserted into incisions and tears. 
\item The points within each set of correspondences are merged and
allocated positions on the 2D surface. 
\item The triangulation points are then projected onto a sphere
curtailed at latitude of  $\varphi _0$ (Figure~1D). Note that whilst
we present the coordinates of the standard retina in terms of
colatitude and longitude, all internal calculations use latitude and
longitude. Colatitude is converted to latitude by subtracting
90$^{\circ}$. The latitude is estimated on the basis of measurements from
intact retinae of animals of the same age as the retina under
reconstruction. The radius  $R$ of the sphere is determined by the
area of the flattened retina and  $\varphi _0$. Points on the rim of
flattened retina are fixed to the rim of the curtailed sphere. 
\item The optimal projection onto the curtailed sphere (Figure~1E) is
inferred by shifting the locations of the vertices on the spherical
surface so as to minimise the energy measure,  $E$, which comprises
the sum of normalised squared differences between lengths of
corresponding connections in flattened and spherical retina ($e_L^2$, see Equation~1) and a penalty term that prevents the
triangles from flipping over:
\begin{equation}
E=\frac 1{2|\mathcal{C}|\overline L}\underset{i\in\mathcal{C}}{\sum
}\frac{\left(l_i-L_i\right)^2}{L_i}+\beta \underset{j\in\mathcal{T}}{\sum
}\left(A_j/\overline A\right)^{\nu }f\left(a_j/A_j\right)
\end{equation}
 where  $L_i$ and  $l_i$ are lengths of corresponding edges 
$i\in\mathcal{C}$ in the flattened and spherical retina respectively, 
$\overline L$ is the mean length of an edge,  $|\mathcal{C}|$ is the number of
edges,  $f$ is a penalty function to be defined below, and  $A_j$ and
 $a_j$ are signed areas of corresponding triangles  $j\in\mathcal{T}$ in the
flattened and spherical retina and  $\overline A$ is the mean of 
$A_j$. The parameter  $\beta $ controls the relative contribution of
the area penalty, and  $\nu $ controls how much larger triangles are
penalised than smaller ones; they are set as described below. The
signed area  $a_j$ is positive for triangles in correct orientation,
but negative for flipped triangles. The penalty function  $f$ is a
piecewise, smooth function that increases with negative arguments:
\begin{equation}
f\left(x\right)=\left\{\begin{matrix}-\left(x-x_0\right)/2&x<0\\\frac
1{2x_0}\left(x-x_0\right)^2&0<x<x_0\\0&x>x_0\end{matrix}\right.
\end{equation}
 The parameter  $x_0$ is set at 0.5. There is thus no penalty
unless triangles have been squashed to less than 50\% of their size
in the flattened retina. The length of each edge  $l_i$ is computed
from the formula for the central angle between its vertices:
\begin{equation}
l\left(\varphi _1,\lambda _1,\varphi _2,\lambda
_2\right)=R\arccos \left(\cos \varphi _1\cos \varphi
_2\cos \left(\lambda _1-\lambda _2\right)+\sin \varphi
_1\sin \varphi _2\right)
\end{equation}
  where  $\varphi _1$ and  $\varphi _2$ are the latitudes of
the vertices and  $\lambda _1$ and  $\lambda _2$ are the longitudes.
The derivatives of  $E$ with respect to  $\varphi _i$ and  $\lambda
_i$ are computed and used to minimise  $E$.  Optimisation proceeds
by: (i)~turning off the area penalty  $\beta =0$ and using the FIRE
structural relaxation algorithm \cite{BitzekEtal2006};  (ii)~starting
from this configuration, setting   $\beta =8$ and  $\nu =1$ and again
using FIRE to minimise the energy, which removes a number of the
flipped triangles, dealing preferentially with the biggest; and
(iii)~minimising using the BFGS quasi-Newton method (as implemented
in the R optim function) with  $\beta =8$  and  $\nu =0.5$ to refine
the optimisation. This procedure was arrived at after trials on a
corpus of over 200 marked-up retinae.
\item The locations of data points and landmarks on the sphere are
determined (Figure 1F). To do this, for each data point the
barycentric coordinates within its containing triangle in the flat
representation are determined. The location of the point on the
sphere is then found by projecting the point to the same set of
barycentric coordinates in the corresponding triangle on the sphere.
From this location in 3D space, represented in Cartesian coordinates,
the spherical coordinates of the point are determined by projection
of a line from the centre of the sphere through the point to the
line's intersection with the sphere. This allows plotting of points
in a polar representation. The procedure can be used in reverse to
infer the locations of lines of latitude and longitude in the flat
retina.
\end{enumerate}
The procedure of reconstructing a retina takes around 1 minute (range
20 seconds to 11 minutes) on a Dell Optiplex 990 with an i7-2600 CPU
running Scientific Linux 5.  

\subsection*{Projections used to display of reconstructed data}
The data on the sphere can be projected onto a polar or azimuthal
equidistant plot in the plane with coordinates  $(\rho ,\lambda )$
where the radius  $\rho =\pi /2+\varphi $. Alternatively the data can
be plotted in an azimuthal equal area (Lambert) projection \cite{FisherEtal1987}, in which area is preserved in the sense that equal areas
on the sphere project to equal areas on the plane of projection,
achieved by setting  $\rho =\sqrt{2(1+\sin \varphi )}$.  Finally
the data can be plotted in an azimuthal conformal (Wulff) projection
in which angles are preserved; this is achieved by setting   $\rho
=\text{tan}\left(\pi /4+\varphi /2\right)$. 

For data transformed into visuotopic space (see later), we used the
sinusoidal orthographic projections. The sinusoidal projection, which
projects the entire globe onto the plane and preserves area, is given
by  $x=(\lambda -\lambda _0)\cos \varphi ,y=\varphi $, where 
$\lambda _0$ is the longitude at the centre of the projection \cite{WolframSinusoidal2012}. The orthographic projection, which gives a
perspective view of one side of the globe, is given by 
$x=\cos \varphi \sin (\lambda -\lambda
_0),y=\cos \varphi _0\sin \varphi -\sin \varphi
_0\cos \varphi \cos (\lambda -\lambda _0)$, where  the
projection is centred on   $(\varphi _{0,}\lambda _0)$ \cite{WolframOrthographic2012}.

By convention, the polar plots are viewed as though the animal is
facing towards the observer. This means that when plotting a retina
from a right eye, the nasal pole on the right and N, D, T, V are in
anticlockwise order; for a retina from a left eye, nasal is on the
left and N, D, T, V are in clockwise order. The longitude of a point
was defined so that 0$^{\circ}$ is always at the right of the plot and
90$^{\circ}$ at top. This means that for a right eye the poles correspond
to the longitudes as follows: N, 0$^{\circ}$; D, 90$^{\circ}$; T, 180$^{\circ}$;
V, 270$^{\circ}$. For a left eye the longitudes of the nasal and temporal
poles are interchanged: N, 180$^{\circ}$; D, 90$^{\circ}$, T, 0$^{\circ}$; V,
270$^{\circ}$.

\subsection*{Analysis of reconstructed
data}
\paragraph{Mean location of data points}
The Karcher mean of a set of points on the sphere \cite{Karcher1977,HeoSmall2006} is defined as the point  $(\overline{\varphi
},\overline{\lambda })$ that has a minimal sum of squared distances
to the set of points  $(\varphi _i,\lambda _i)$:
\begin{equation}
(\overline{\varphi },\overline{\lambda })=\text{arg min}_{(\varphi
,\lambda )}\sum _{i=1}^Nl^2(\varphi ,\lambda ,\varphi _i,\lambda _i)
\end{equation}
where the function  $l$ is defined in Equation 5. 

\paragraph{Kernel density estimates}
We use Diggle and Fisher's method for producing kernel density plots
of data points lying on a sphere \cite{DiggleFisher1985}. The
underlying kernel function is the Fisherian density \cite{Fisher1953}:
\begin{equation}
F(\varphi ,\lambda ,\varphi _0,\lambda _0;\kappa )=\frac{\kappa }{4\pi
\cosh \kappa }\exp (\kappa \cos l(\varphi ,\lambda
,\varphi _0,\lambda _0))=\frac{\kappa }{4\pi \cosh \kappa
}\exp (\kappa \vec r\cdot \vec r_0)
\end{equation}
where  $\kappa $ is the ``concentration'' parameter and   $\vec r$ and
 $\vec r_0$ are the Cartesian coordinates of points with spherical
coordinates  $(\varphi ,\lambda )$ and  $(\varphi _0,\lambda _0)$ on
the unit sphere. For large  $\kappa $ the Fisherian density is
approximately a bivariate Gaussian with variance  $\sigma ^2=\kappa
^{-1}$ \cite{DiggleFisher1985}. The density at a location  $(\varphi
,\lambda )$ is defined as:
\begin{equation}
\hat f(\varphi ,\lambda ;\kappa )=\frac 1 n\sum _{i=1}^nF(\varphi
,\lambda ,\varphi _i,\lambda _i;\kappa )
\end{equation}
 $\kappa $  is chosen so as to maximise a cross-validated
log-likelihood:
\begin{equation}
L(\kappa )=\sum _{i=1}^n\log \{\hat f_i(\varphi _i,\lambda
_i;\kappa )\}
\end{equation}
where  $\hat f_i(\varphi ,\lambda ;\kappa )$ is the kernel density
estimate calculated from all but the  $i$th data point. 

\paragraph{Kernel regression estimates}
In kernel regression, we have values  $y_i$ at each data point 
$(\varphi _i,\lambda _i)$. We adapt the Nadaraya-Watson estimator \cite{Nadaraya1964,Watson1964} to the Fisher density and the spherical
coordinates:
\begin{equation}
\hat y(\varphi ,\lambda ;\kappa )=\frac{\sum _{i=1}^nF(\varphi
,\lambda ,\varphi _i,\lambda _i;\kappa )y_i}{\sum _{i=1}^nF(\varphi
,\lambda ,\varphi _i,\lambda _i;\kappa )}
\end{equation}
The concentration parameter  $\kappa $ is set by minimising the summed
squared error:
\begin{equation}
\sum _{i=1}^n(y_i-\hat y_i(\varphi _i,\lambda _i;\kappa ))^2
\end{equation}
 where  $\hat y_i(\varphi _i,\lambda _i;\kappa )$ is the kernel
regression estimate of  $y_i$ calculated from all but the  $i$th
data point. 

\paragraph{Contour analysis}
The kernel estimates can be represented using a contour plot. This is
produced by the standard method \cite{FisherEtal1987} of laying a 100
by 100 Cartesian grid over the azimuthal area-preserving (Lambert)
projection, finding the spherical coordinates corresponding of each
point on the grid, and then computing the density estimate at each
point. A standard contouring algorithm is then used to plot the
contours. There are a number of methods for determining contour
heights \cite{FisherEtal1987}. We chose to define them in terms of the
amount of probability mass that they exclude; e.g. the 5\% contour
excludes 5\% of the probability mass. The area contained within each
contour can be determined simply by finding the area within the
contour in the area-preserving (Lambert) projection \cite{FisherEtal1987}. 

\subsection*{Mapping of retina onto visual space}
In previous work of the mapping of visual space on the superior
colliculus \cite{DragerHubel1976}, visual space has been described
in terms of azimuth  $\theta $  and elevation  $\alpha $  with
respect to the long axis of the mouse. Visual space is mapped onto
the retina via the optical system of the eye. This mapping depends on
the angle that a ray makes with the optic axis, which we define as
being oriented at azimuth  $\theta _0$  and  elevation  $\alpha _0$.

In order to map the eye onto visual space, we consider a reference
frame in which the $z${}-axis is vertical, the
$x${}-axis is pointing forwards along the long axis of the
mouse and the $y${}-axis is pointing rightwards, as viewed by
an observer facing the animal. Notionally, the optic axis of the eye
starts off vertical, and a sequence of rotations is undertaken in
order to move the eye to its correct orientation. First the
projection of rays from the retina to visual space is determined
using the approximate optics defined in the main text. Second, the
coordinates of points are converted to Cartesian coordinates. Third,
the coordinates are rotated about the $x${}-axis by 
$-90+\alpha _0$ degrees. This ensures that the optic axis is oriented
at the elevation  $\alpha $ above the horizontal, but the direction
of its azimuth will be  $-90^{\circ }$. The points on the eye are
then rotated about the original $z${}-axis by  $90+\theta _0$
degrees giving the optic axis an azimuth of  $\theta _0$, where
positive values are to the right, as viewed from a point on the
positive $x${}-axis. Finally the Cartesian coordinates are
transformed back to spherical coordinates in the original reference
frame.

\subsection*{Acknowledgements}
This work was supported by a Programme Grant from the Wellcome Trust
(G083305) to DJW, IDT,  Stephen Eglen and Uwe Drescher. DL was
supported by a Medical Research Council (UK) Studentship. The authors
would like to thank Andrew Lowe, Stephen Eglen, Johannes JJ Hjorth,
and Michael Herrmann for their very helpful comments throughout this
work. We would also like to thank Nicholas Sabicki for helping with
the analysis of LGN projections and Henry Eynon-Lewis for assistance
with dissections for muscle-insertion marking and S-opsin staining.
\clearpage\section*{Figure Legends}
\begin{figure}[!ht]\caption{
\textbf{Overview of the method. A,} The raw data: a retinal
outline from an adult mouse (black), two types of data points (red
and green circles) from paired injections into the superior
colliculus and a landmark (blue line). \textbf{B, }Retinal outline
with nasal pole (N) and incisions marked up. Each pair of dark cyan
lines connects the vertices and apex of the four tears. \textbf{C,}
The outline after triangulation (shown by grey lines) and stitching,
indicated by cyan lines between corresponding points on the tears.
\textbf{Di,} The initial projection of the triangulated and stitched
outline onto a curtailed sphere. The strain of each edge is
represented on a colour scale with blue indicating compression and
red expansion. Tears are shown in cyan. \textbf{Dii,} The strain
plotted on the flat outline with lines of latitude and longitude
superposed. \textbf{Ei,ii,} As Di,ii but after optimisation of the
mapping. \textbf{Fi,} The data represented on a polar plot of the
reconstructed retina. Mean locations of the different types of data
points are indicated by  diamonds. The nasal (N), dorsal (D),
temporal (T) and ventral (V) poles are indicated. Tears are shown in
cyan. \textbf{Fii,} Data plotted on the flat representation, with
lines of latitude and longitude superposed. All scale bars are 1\,mm.
}\end{figure}

\begin{figure}[!ht]\caption{
\textbf{Examples of reconstructed retinae. A--D,} An example
of a reconstruction of an adult retina with low deformation measure
$e_\mathrm{L}=0.038$. \textbf{A,} Plot of length of edge
on the sphere versus length of edge on the flat retina. Red indicates
an edge that has expanded and blue a edge that has been compressed.
\textbf{B,} The log strain  $\ln l_i/L_i$ indicated using the
same colour scheme on the flat retina. \textbf{C,}\textbf{~}The flat
representation of lines of latitude and longitude with the optic disc
(blue). \textbf{D,} The azimuthal equidistant (polar) representation
showing the locations of the incisions and tears (cyan) and the
location of the optic disc (blue). \textbf{E--H,} An example of a
reconstruction of a P0 retina with high deformation energy
$e_\mathrm{L}=0.118$. Meaning of E--H same as for
corresponding panel in A--D. All scale bars are 1\,mm.
}\end{figure}

\begin{figure}[!ht]\caption{
\textbf{Deformation of reconstructions and the effect of rim
angle. A,} Histogram of the reconstruction error measure
$e_\mathrm{L}$ obtained from 288 successfully
reconstructed retinae.  \textbf{B,} Relationship between deformation
measure and age. ``A'' indicates adult animals.
\textbf{C,}\textbf{~}Schematic diagram of eye, indicating the
measurements   $d_{\mathrm{e}}$ and  $d_{\mathrm{r}}$ made on mouse eyes at different
stages of development, and the rim angle  $\varphi _0$ derived from
these measurements. Note that rim angle is measured from the optic
pole~(*). \textbf{D,}\textbf{~}Rim colatitude  $\hat{\varphi }_0$
that minimises reconstruction error  versus the rim angle  $\varphi
_0$ determined  from eye measurements. Solid line shows equality and
grey lines indicate  $\pm 10^{\circ }$ and  $\pm 20^{\circ }$ from
equality. \textbf{E,}\textbf{~}Minimum reconstruction error  $\hat
e_{\text L}$ obtained by optimising rim angle versus reconstruction
error  $e_{\text L}$ obtained when using the rim angle from eye
measurements. Solid line indicates equality.
}\end{figure}

\begin{figure}[!ht]\caption{
\textbf{}\textbf{Estimation of reconstruction error using
optic disc location. }\textbf{A,} Inferred positions of optic discs
from 72 adult reconstructed retinae plotted on the same polar
projection. The colatitude and longitude of the Karcher mean is
(3.7$^{\circ}$, 95.4$^{\circ}$). The standard deviation in the angular
displacement from the mean is  7.4$^{\circ}$. \textbf{B,} The same data
plotted on a larger scale. \textbf{C, }The relationship between the
deformation of the reconstruction and distance  $\varepsilon
_{\text{OD}}$  of the inferred optic disc from the population mean.
There was a significant correlation between the two ($R^2=0.35,p<0.01$)\textit{.}
}\end{figure}

\begin{figure}[!ht]\caption{
\textbf{Measurement of the ipsilateral projection.
}\textbf{A, }Schematic illustrating the retinal label resulting from
bilateral injections of red and green dye into the dLGN.\textbf{ B,
}\textstylecaptiontitle{\textmd{Flat-mounted retina
with}}\textstylecaptiontitle{\textmd{ }}label resulting from
bilateral injections of Fluoro-Ruby and Fluoro-Emerald into left and
right dLGN, respectively.\textbf{ C-D,
}\textstylecaptiontitle{\textmd{Azimuthal equilateral projection of
reconstructed retina in B. Isodensity contours for 5\%, 25\%, 50\%,
75\% \& 95\% are plotted using the KDE estimates for fully sampled
retinae (C) or KR estimates for partially sampled retinae (D). Blue
diamond is the peak density and red (C) or green (D) diamond is the
Karcher mean. Yellow circle is the optic disc.
}}\textstylecaptiontitle{E,}\textstylecaptiontitle{\textmd{ Composite
plot with label from ipsilateral injections
(}}\textstylecaptiontitle{\textmd{$n$}}\textstylecaptiontitle{\textmd{=7).
Black dashed lines represent the median angle from the optic axis to
the peripheral }}\textstylecaptiontitle{\textmd{edges of the 5\%
isodensity contour. Coloured diamonds represent the Karcher means of
the label in the individual retinae and large coloured circles are
the optic discs for the individual retinae. White square and circle
represent the average Karcher mean. The central dashed angle
represents the median central edge of the 5\% isodensity contour.
}}\textstylecaptiontitle{F,}\textstylecaptiontitle{\textmd{ Mean
distances from either optic disc or optic axis of reconstructed
retinae to the central edge of the 5\% isodensity contour along a
line passing through the Karcher mean of the
label.}}\textstylecaptiontitle{ G,}\textstylecaptiontitle{\textmd{
The extent of the ipsilateral label and the distance beyond the
horizontal and vertical axes. Grid spacing is 20$^{\circ}$. In F-G, line
represents the mean and error bars are standard error of the mean.
Scale bar in C \& D is
1}}\textstylecaptiontitle{\textmd{~}}\textstylecaptiontitle{\textmd{mm.}}\textstylecaptiontitle{
}\textbf{ }
}\end{figure}

\begin{figure}[!ht]\caption{
\textbf{Alignment of the binocular zone in visuotopic
coordinates.} \textbf{A,} Azimuthal equilateral projections in
standard\textit{ retinal }space of left and right retinae with
ipsilateral (upper) and contralateral (lower) label resulting from
bilateral injections of Fluoro-Ruby and Fluoro-Emerald into left and
right dLGN, respectively, of the same mouse. Plots were generated
from stitched 10x epifluorescent images and cell locations detected
using ImageJ. For this figure, we have abandoned the convention of
always plotting nasal retina to the right. \textbf{B,} Schematic
illustrating the approximate projection of \textit{retinal} space
onto \textit{visual} space. When the orientation of the optic axis
(grey line) is optimal, the ipsilateral crescent is projected
entirely to the opposite visual field. Note that due to the
refraction in the lens the visual field the visual field is estimated
to be 180$^{\circ}$ for each eye. \textbf{C-D,} Orthographic projections
in central \textit{visuotopic} space of the two ipsilateral retinae
in A with optic axis (*) at 64$^{\circ}$ azimuth; 22$^{\circ}$ elevation (C)
and with optic disc at 60$^{\circ}$ azimuth; 35$^{\circ}$ elevation
(D).\textbf{ E,} Sinusoidal projection of contralateral retinae in B
with the optic axis (*) at 64$^{\circ}$ azimuth; 22$^{\circ}$ elevation.
Labels N, D, T, V indicate the projection of the corresponding pole
of the retina. Grid spacing is 15$^{\circ}$.
}\end{figure}

\begin{figure}[!ht]\caption{
\textbf{Measurement of muscle insertion angles.} \textbf{A,}
Flat-mounted retina showing stitching and insertions for superior
rectus (red), lateral rectus (green) and inferior rectus (blue).
N~indicates nasal cut. Plots on right represent the distortions
introduced by reconstructing retina (see Figure~2 for explanation).
\textbf{B,} Azimuthal equilateral projection of reconstructed retina
in A. Dashed lines represent vectors connecting muscle insertion
point to the optic disc. \textbf{C,} Muscle insertion points from 17
retinae. Solid black circles represent the optic discs for individual
retinae. Dashed lines represent the line from each individual
insertion point to its respective optic disc. Solid lines are from
the Karcher mean insertion to the Karcher mean location of the optic
disc. Grid Spacing is 15$^{\circ}$.\textbf{ D,} Plot of the angles of the
angles of vectors connecting muscle insertions of Superior Rectus
(SR), Lateral Rectus (LR) and Inferior Rectus (IR) to the individual
optic discs. Bar represents the mean and error-bars are standard
deviation.
}\end{figure}

\begin{figure}[!ht]\caption{
\textbf{Visuotopic axes with respect to
}\textbf{S-opsin}\textbf{ distribution. A,} S-opsin staining in
dorsal, central and ventral retina. Images acquired at 20x
magnification. Scale bar is 100\,{\textmu}m.\textbf{ B-C, }S-opsin
distribution for right (A) and left (B) eyes plotted in orthographic
projection centred on optic axis (*) at 22$^{\circ}$ elevation and
64$^{\circ}$ azimuth. Bottom left plot is flat-mounted retina. Bottom
right plot is azimuthal equilateral plot. Plots were generated from
stitched 10x epifluorescent images and cell locations detected using
ImageJ. Again our normal convention of showing nasal to the left has
been relaxed. Scale bar is 1\,mm.\textbf{ D,} The average offset of
the S-opsin density-transition from the horizontal meridian in
central and peripheral visual field ($n=9$). Error bars are
SEM. \textbf{E,} S-opsin distribution for both eyes plotted in a
sinusoidal projection with same optic axis (*) as in C. Yellow
outline is edge of left retina; red outline is edge of right retina. 
Labels N, D, T, V indicate the projection of the corresponding pole
of the retina. Grid spacing is 15$^{\circ}$. 
}\end{figure}

\clearpage\section*{Tables}
\begin{table}[!ht]\caption{
\textstylecaptiontitle{}\textstyletextbf{Measurements of
mouse eyes at various stages of development.}
}
\begin{tabular}{|l|r@{}l|r@{}l|r@{}l|}


\hline
Age &
\raggedleft  $d_{\mathrm{e}}$  ({\textmu}m)  &
 &
 $d_{\mathrm{r}}$  ({\textmu}m)  &
 &
 $\varphi _0$ ($^{\circ}$) &
\\\hline
P0  &
\raggedleft 1632 $\pm $   &
17 &
1308 $\pm $   &
17 &
127.13 $\pm $   &
1.92\\\hline
P2  &
\raggedleft 2146 $\pm $   &
13 &
1780 $\pm $   &
29 &
131.20 $\pm $   &
2.20\\\hline
P4  &
\raggedleft 2250 $\pm $   &
9  &
1857 $\pm $   &
20 &
130.58 $\pm $   &
1.40\\\hline
P6  &
\raggedleft 2450 $\pm $   &
41 &
1963 $\pm $   &
25 &
127.02 $\pm $   &
2.41\\\hline
P8  &
\raggedleft 2646 $\pm $   &
1  &
2088 $\pm $   &
34 &
125.31 $\pm $   &
1.81\\\hline
P12  &
\raggedleft 2786 $\pm $   &
15 &
2212 $\pm $   &
65 &
126.03 $\pm $   &
3.37\\\hline
P16  &
\raggedleft 2808 $\pm $   &
17 &
2043 $\pm $   &
17 &
117.10 $\pm $   &
0.96\\\hline
P22  &
\raggedleft 2958 $\pm $   &
35 &
2117 $\pm $   &
44 &
115.57 $\pm $   &
2.17\\\hline
P64  &
\raggedleft 3160 $\pm $   &
56 &
2161 $\pm $   &
30 &
111.56 $\pm $   &
1.89\\\hline
\end{tabular}
\begin{flushleft}
The distance from the back of the eye to the surface of the cornea 
$d_{\mathrm{e}}$, the perpendicular distance from the back of the eye to the rim
of the retina  $d_{\mathrm{r}}$ and the rim colatitude  $\varphi _0$ derived
from this. Each measurement is averaged over the right and left eyes
of two different animals, i.e. over four eyes in total.
\end{flushleft}\end{table}

\bibliography{nmf_morph}
\end{document}

